%       Process with LaTeX (Tested with LaTeX2e <2002/06/01>)

\documentclass[twocolumn]{article}
% This file expects to be \input by somebody's preamble.
% It defines the trademark and registered trademark symbols.
% They were designed in 11 point type with lots of empirical constants.
% Everything is expressed in `ex' units, with the intention of making it
% work in all sizes.  Remarkably, it has passed tests at 10 and 12 points.
%
% Define the trademark symbol
\newcommand{\tm}{\setlength{\unitlength}{1ex}%
\begin{picture}(2.6,1)\put(0,.7){\tiny TM}\end{picture}}
%
% Define the Registered trademark symbol
\newcommand{\rtm}{\setlength{\unitlength}{1ex}%
\begin{picture}(1.35,1)\put(.175,.7){\tiny R}\put(.675,1.1){\circle{1.25}}%
\end{picture}}

\usepackage{aitech}
\usepackage{atldoc}
\usepackage{alltt}
\newfont{\atlsub}{cmssi17}
\newfont{\atlname}{cmssbx10 at 11pt}
\title{
ATLAST\\
\vspace{3pt}
$\overline{\makebox{\underline{\atlsub Autodesk Threaded Language Application
System Toolkit}}}$
}
\author{John Walker}
\date{Revision 1 by Duff Kurland---November 20, 1990}
\abstract{
        Open, programmable products are superior to and displace even
        the best designed closed applications.  A threaded language,
        implemented in a single portable C file, allows virtually any
        program, existing or newly developed, to be made programmable,
        extensible, and open to user enhancement.
}

\begin{document}
\maketitle

\newcommand{\atlast}{\underline{\underline{\sc Atlast}}}
\hyphenation{Auto-CAD Auto-LISP}

\dcap{Y}OU'D THINK WE'D HAVE LEARNED by now.  It was Autodesk's
strategy for AutoCAD\rtm\ from inception that it should be an open,
extensible system.  We waged a five-year uphill battle to bring such a
heretical idea to eventual triumph.  Today, virtually every industry
analyst agrees that AutoCAD's open architecture was, more than any
other single aspect of its design, responsible for its success and the
success that Autodesk has experienced.

And yet, even today, we write program after program that is
closed---that its users cannot program---that admits of no extensions
without our adding to its source code.  If we believe intellectually,
from a sound understanding of the economic incentives in the
marketplace, that open systems are better, and have confirmed this
supposition with the success of AutoCAD, then the only question that
remains is {\em why?}  Why not make every program an open program?

Well, because it's {\em hard\/}!  Writing a closed program has
traditionally been much less work at every stage of the development
cycle: easier to design, less code to write, simpler documentation,
and far fewer considerations in the test phase.  In addition, closed
products are believed to be less demanding of support,
although I'll argue later that this assumption may be incorrect.

\subsection{The painful path to programmability}

Most programs start out as nonprogrammable, closed applications, then
painfully claw their way to programmability through the introduction
of a limited script or macro facility, succeeded by an increasingly
comprehensive interpretive macro language which grows like topsy and
without a coherent design as user demands upon it grow.  Finally,
perhaps, the program is outfitted with bindings to existing languages
such as C\@.

An alternative to this is adopting a standard language as the macro
language for a product.  After our initial foray into the awful menu
macro language that still burdens us, AutoCAD took this
approach, integrating David Betz' XLISP, a simple Lisp interpreter
which was subsequently extended by Autodesk to add floating point,
many additional Common Lisp functions, and, eventually, access to the
AutoCAD database.

This approach has many attractions.  First, choosing a standard
language allows users to avail themselves of existing books and
training resources to learn its basics.  The developer of a
dedicated macro language must create all this material from scratch.
Second, an interpretive language, where all programs are represented
in ASCII code, is inherently portable across computers and operating
systems.  Once the interpreter is gotten to work on a new system, all
the programs it supports are pretty much guaranteed to work.  Third,
most existing languages have evolved to the point that most of the
rough edges have been taken off their design.  Extending an existing
language along the lines laid down by its designers is much
less likely to result in an incomprehensible disaster than growing an
ad-hoc macro language feature by neat-o feature.

Unfortunately, interpreters are {\em slow}, {\em slow}, {\em slow}.
A simple calculation of the number of instructions of overhead per
instruction that furthers the execution of the program quickly
demonstrates that no interpreter is suitable for serious computation.
As long as the interpreter is deployed in the role of a macro
language, this may not be a substantial consideration.  Most early
AutoLISP\rtm\ programs, for example, spent most of their time submitting
commands to AutoCAD with the {\tt (command)} function.  The execution
time of the program was overwhelmingly dominated by the time AutoCAD
took to perform the commands, not the time AutoLISP spent constructing
and submitting them.  However, as soon as applications tried to do
substantial computation, for example the parametric object
calculations in AutoCAD AEC, the overhead of AutoLISP became a
crushing burden, verging on intolerable.  The obvious alternative
was to provide a compiled language.  But that, too, has its problems.

\section{Introducing \atlast}

\atlast\tm\ is a toolkit that makes applications
programmable.  Deliberately designed to be easy to integrate both
into existing programs and newly-developed ones, \atlast\ provides
any program that incorporates it most of the benefits of programmability
with very little explicit effort on the part of the developer.
Indeed, once you begin to ``think \atlast'' as part of the design
cycle, you'll probably find that the way you design and build programs
changes substantially.  I'm coming to think of \atlast\ as the
``monster that feeds on programs,'' because including it in a program
tends to shrink the amount of special-purpose code that would
otherwise have to be written while resulting in finished applications
that are open, extensible, and more easily adapted to other operating
environments such as the event driven paradigm.

The idea of a portable toolkit, integrated into a wide variety of
products, all of which thereby share a common programming language
seems obvious once you consider its advantages.  It's surprising
that such packages aren't commonplace in the industry.  In fact,
the only true antecedent to \atlast\ I've encountered in my whole twisted
path through this industry was the universal macro package developed
in the mid 1970's by Kern Sibbald and Ben Cranston at the University
of Maryland.  That package, implemented on Univac mainframes, provided
a common macro language shared by a wide variety of University of
Maryland utilities, including a text editor, debugger, file dumper,
and typesetting language.  While \atlast\ is entirely different in
structure and operation from the Maryland package, which was an
interpretive string language, the concept of a cross-product macro
language and appreciation of the benefits to be had from such a
package are directly traceable to those roots.

So what {\em is} \atlast?  Well\ldots it's FORTH, more or less.
Now I'm well aware that the mere mention of FORTH stimulates a violent
immune reaction in many people second, perhaps, only to that induced by the
utterance of the dreaded word ``LISP\@.''  Indeed, more that 12 years
after my first serious encounter with FORTH, I am only now coming to
feel that I am truly beginning to ``get it''---to understand what it's
really about, what its true strengths (and weaknesses) are, and to what
problems it can offer uniquely effective solutions.  PostScript had
a lot to do with my coming to re-examine FORTH, as did my failed
attempt in early 1988 to separate AutoCAD's user interface from the
geometry engine.  That project, The Leto Protocol, ended with my
concluding that to succeed: to create an interface that would not
grow to unbounded size, bewildering complexity, and glacial
performance, it would be necessary to embed programmability within the
core---to provide a set of primitives that could be composed, by the
user interface module, into higher-level operators that could be
invoked across the link between the two components.  This
programmability would, of course, have to be in a portable form and
not involve linking user code into the AutoCAD core.

In looking for parallels to the problem I faced, PostScript seemed
similarly motivated and reasonably effective in accomplishing its
goals.  (One can certainly attack PostScript on performance, although
I suspect its performance problems stem more from the underlying
execution speed of the graphics primitives and the inefficient ASCII
representation of input than any inherent aspect of the language.)
Certainly PostScript blew away its competitors, such as Impress and
DDL, almost without taking notice of them.  Further, it seemed
apparent that PostScript's success was another example in the long
list of open, programmable products that triumphed over ``more
comprehensive'' but non-extensible ones.

Looking at PostScript inevitably brings one back to the language
that inspired it, FORTH\@.  Although FORTH has a reputation for
obscurity and seems to attract an unusually high percentage of flaky
adherents, it has many attributes that recommend it as a candidate for
a portable tool to make any application programmable.

\paragraph{It is small.} A minimal implementation of FORTH is a tiny
thing indeed, since most of the language can be defined in itself,
using only a small number of fundamental primitives.  Even a rich
implementation, with extensions such as floating point and
mathematical functions, strings, file I/O, compiler writing
facilities, user-defined objects, arrays, debugging tools, and
runtime instrumentation, is still on the order of one fifth the number
of source lines of a Lisp interpreter with far fewer built-in
functions, and occupies less than of 70\% the object code size.  Runtime
data memory requirements are a tiny fraction (often one or two percent) of
those required by Lisp, and frequently substantially less that
compiled languages such as C\@.  It's kind of startling to discover
that an entire interpretive and compiled language, including floating
point, all the math functions of C, file I/O, strings, etc., can be
built, in large model, into a DOS executable of 50964 bytes.  It can.

\paragraph{It is fast.}  Because it is a threaded language, execution
of programs consists not of source level interpretation but simple
memory loads and indirect jumps.  Even for compute-bound code, the
speed penalty compared to true compilers is often in the range of 5 to
8.
While this may seem a serious price to pay, bear in
mind that tokenising Lisp interpreters often exhibit speed penalties
of between 60 and 70 to 1 on similar code, and source-level
interpreters, such as the macro languages found in many application
programs, are often much, much worse than that.  In most programs, the
execution speed of FORTH and compiled code will be essentially
identical, particularly when FORTH is used largely in the role of a
macro language, calling primitives within an application coded in a
compiled language.

\paragraph{It is portable.}  If the implementation rigidly specifies
the memory architecture and data types used (and this can be done with
essentially no sacrifice in speed), FORTH programs can be made 100\%
compatible among implementations.  Programs can be transferred as
ASCII files, universally interchangeable across systems.
Application data types defined in FORTH, using its object creation
facilities, automatically gain the portability of the underlying
data types.

\paragraph{It is easy to extend.}  Because the underlying architecture
is very simple (unlike, for example, that of a Lisp interpreter), any
competent C programmer with a minimum of indoctrination can begin
adding C-coded primitives to a C-implemented FORTH within hours.
These C primitives will run at full speed, yet be able to be
parameterised, placed in definitions, used in loops, etc., from any
FORTH construct.  This leads to a different way of building
applications.  Rather than programming the structure and primitives as a
unified process, one builds the application-unique primitives that are
needed, tests them interactively as they are built, then assembles the
application with glue code written either in FORTH or C
depending upon considerations of efficiency, security, and the extent
to which one wishes to make the underlying primitives visible to and
accessible by the user.  Unlike conventional program development
processes, these considerations are not yes-or-no decisions but, for
the most part, continua along which the product may be positioned at
the point desired and subsequently adjusted based upon market feedback.

\paragraph{It is interactive.}  While most portions of a FORTH program
are compiled into a form equally compact and comparable in execution
speed to machine code, direct user interaction can always be furnished
simply by providing a connection from the user's keyboard to the
interpreter (or conversely, blocked by denying the user that access).
That such interactivity expedites program development compared to the
normal edit, compile, link, debug cycle is well known.  That FORTH can
provide it without sacrificing execution speed is one of its major
attractions.

\paragraph{It supports multiple operating paradigms.}  Once the
technique of encapsulating the functionality of a product in
primitives accessible from the FORTH environment is mastered, it is
possible to build programs in which the core facilities (for example,
database access, geometric calculations, graphical display of results,
calculating mass properties) can be composed into sequences that can
be invoked from a program, called interactively from a command line,
triggered by a menu selection or pick of a button in a dialogue, or
virtually any other form of interaction imaginable.  Further, since
any stimulus that affects the program simply
executes a FORTH word, and such words can be easily redefined with a
small amount of FORTH text, any of these operating modes can be
rendered programmable by the implementor, third party developer, or
user, at the discretion of the designer.

\paragraph{It is surprisingly modern.}  Although FORTH appears to be
an artifact of the bygone days of 64K computers and teletype machines,
many of its concepts, viewed through contemporary eyes, are remarkably
up to date.  For example, few languages share its ability to define
new fundamental data types, along with methods that operate upon
them.  The multiple dictionary facility of FORTH permits one to create
objects that inherit, by default, properties of their parents, and to
implement such structures in an efficient manner.

\subsection{\atlast\ and FORTH}

All of these advantages do not erase some substantial shortcomings of
FORTH, particularly in the modern programming environment.  In
defining \atlast , I have attempted to conform to FORTH wherever possible,
without compromising my overall goal of creating a system that would
allow a developer to factor out the programmability from an
application and hand it to a standard module to manage, precisely as
C programmers delegate I/O and mathematical function evaluation to
library routines provided for those purposes.

\atlast\ is based on the FORTH-83 standard and incorporates many of the
optional extensions and supplementary words defined in that standard.
Once the basic differences between FORTH and \atlast\ have been
mastered, one can use a FORTH reference manual for most user-level
\atlast\ programming tasks.  The major differences between FORTH-83 and
\atlast\ are as follows.

\paragraph{Integers are 32 bits.}  To bring forth another language
burdened with 16 bit integers in the year 1990 is, to my mind,
unthinkable.  We are rapidly entering an era where the vast
majority of C language environments agree that the {\tt int} type is
32 bits, and applications may be expected to rapidly conform to this
standard.  Consequently, in \atlast, all integers are 32 bits and no
{\tt short} data type is provided.  Note that this does not imply
incompatibility with C environments with 16 bit {\tt int}s---\atlast\
works perfectly with Turbo C on MS-DOS and Microsoft C on OS/2, for
example, because all integers are explicitly declared as {\tt long}.

\paragraph{Identifiers are arbitrary length.}  In \atlast , you need
not struggle with the tradeoff between memory efficiency and
uniqueness of identifiers that plagues the FORTH programmer.
Identifiers are limited in length only to the size of the built-in
token assembly buffer, which defaults to 128 characters, and
all characters are significant.  Again, this change brings \atlast\ more
closely into conformance with contemporary language designs.
To implement this change, symbol names were moved from the heap into
dynamically allocated buffers, taking advantage of the underlying C
runtime environment.  This makes the task of adjusting heap size
easier (and changes some of the arcana of programs that fiddle with
the low-level structure of the system, but everything you could do in
FORTH, you can do in \atlast, albeit in a slightly different way).

\paragraph{Floating point is supported.}  Floating point constants,
variables, operators, scanning and formatting facilities, and a rich
set of mathematical functions are provided as
primitives (which can be turned off at compile time, if not needed).
Compatibly with C, the default floating point type is 64 bit
C {\tt double} precision numbers.  The only assumption made by \atlast\
about floating point format is that a floating point number is twice
the size of an integer.  The rational number facilities of FORTH are
not provided in \atlast .

\paragraph{Strings are supported.}  Strings are supported at a much
higher level in \atlast\ than in FORTH\@.  String literals are provided
in a general and explicit manner using the C syntax for escaping
special characters.  A rich set of string processing functions which
closely follow those of C are provided ({\tt STRCPY}, {\tt STRCAT},
{\tt STRLEN}\ldots ).  A mechanism of cyclically allocated temporary
string buffers provides more flexible manipulation of strings in
interactive input.  Strings continue to follow the pointer and buffer
model used by both C and FORTH\@.  String-intensive programs should
run at about the same speed as their equivalents in C or FORTH\@.

\paragraph{Debugging facilities are provided.}  \atlast\ can be
configured at compile time with as much or as little error checking
and debugging support as is appropriate for the application in which
it is being integrated and the development status of that product.
During development and test, one can configure \atlast\ with an
optional {\tt TRACE} that follows program execution primitive by
primitive, a {\tt WALKBACK} that prints the active word stack when an
error is detected, precise overflow and underflow checking of both the
evaluation and return stacks, and close to bulletproof pointer
checking that catches attempts to load or store outside the designated
heap area.  Although sufficiently crafty programs can still crash
\atlast , errors that slip past the checking and wreak havoc are
extremely rare, even in unprotected environments such as MS-DOS\@.
This, combined with the fundamental interactivity of \atlast ,
makes for a friendly debugging environment.  All the runtime error
checking can be disabled to reduce memory and execution time overhead,
when and where appropriate.

\paragraph{File I/O follows C and Unix conventions.}  FORTH was
developed before the age of standard operating systems; in its early
days, it {\em was} the operating system of many of the minicomputers
which ran it.  Now that the Unix file system interface has become a {\em de
facto} industry standard, \atlast\ conforms to that model of file
system operation.  {\tt FILE} variables correspond to C language file
descriptors, and a familiar set of primitives such as {\tt FOPEN},
{\tt FCLOSE}, {\tt FREAD}, {\tt FSEEK}, etc., are used in the same
manner as in C\@.  Line-level I/O is provided as well, offering
AutoCAD-compatible automatic recognition of ASCII files written with any
of the current end of line conventions.

\paragraph{Extensive support for embedding is provided.}  Unlike
FORTH, \atlast\ is intended to be invisibly embedded within application
programs.  Other than providing a common framework for programmability
and extension, the application continues to ``look like'' itself, not
like \atlast\ or FORTH\@.  Thus, \atlast\ is not ``in control'' in the
sense that the main loop of a FORTH system is; it is a slave, called
by the application at appropriate times.  Accomplishing this required
inverting the control structure from that of a typical FORTH system
and providing a comprehensive set of C callable linkages by which the
application communicates with \atlast .  In addition, primitives are
provided which aid in tuning \atlast\ to the precise needs of the host
program.  The developer can monitor memory usage, note which
primitives are used and which are not, and configure a custom version
of \atlast\ ideally suited to the needs and environment of the host
program.

\section{A note on what follows}

In order to illustrate \atlast , the balance of this paper employs
numerous sample programs and fragments of \atlast\ code.  A reader with
a basic understanding of FORTH should, along with the definitions of
the \atlast\ primitives given at the end of the paper, be able to
figure out what is going on in the examples.  If you've never
encountered FORTH before, the examples may seem little more than
gibberish.  Don't worry---once you get the hang of it, or consult one
of the many excellent FORTH books available (I recommend {\sl
Mastering Forth}, by Anderson and Tracy, New York: Brady
Books/Prentice-Hall, 1984), all will become clear.

Until then, don't be put off by the examples.  Just skim over them
{\em as if\/} you understood them.  You'll still pick up the flavour
of the package, how it integrates with applications, and what you can
do with it.  I'd like to be able to leave my brain and fingers running
overnight and find a complete \atlast\ reference manual that could
stand by itself sitting on my machine the next day.  Alas, I lack
overnight batch capability and have no opportunity to undertake such a
task in prime time at present.  I decided to supply the documentation
in this oddly incomplete form to get the essentials across to those
who can understand it rather than defer the entire effort until I can
complete a hundred pages or so of documentation that largely
duplicates a FORTH reference manual.

\section{Interactive \atlast}

Although \atlast\ is intended to be embedded in application programs,
for learning the language, experimenting with small programs, and
using it as a desk calculator, it's handy to have an interactive
stand-alone version.  The \atlast\ source distribution includes a main
program, {\tt atlmain.c}, that can be linked with \atlast\ to provide
such a utility.  The executable, called {\tt atlast} on Unix and {\tt
ATLAST.EXE} on MS-DOS, is built with all error checking enabled to aid
in program development.

To experiment with \atlast , execute the interactive program with:

{\tt atlast}

You'll be prompted with:

{\tt ->}

as long as \atlast\ is in the interpretive state.  For example, you
might load \atlast\ and experiment with various rational approximations
of $\pi$.

\begin{verbatim}
    % atlast
    -> 22.0 7.0 f/ f.
    3.14286 -> 377.0 120.0 f/ f.
    3.14167 -> ^D
    %
\end{verbatim}

Note that \atlast\ does not explicitly return the carriage after
output; use the {\tt CR} primitive if you wish this done.  Rather than
printing each number and comparing it manually against $\pi$, we can
define a {\em constant} with the value of $\pi$ and a new {\em word}
(or function) that compares a value against it and prints the error
residual.  Here's how we might do that:

\begin{verbatim}
    % atlast
    -> 1.0 atan 4.0 f* 2constant pi
    -> : pierr
    :>   pi f- fabs f. cr
    :> ;
    -> 3.0 pierr
    .141593
    -> 22.0 7.0 f/ pierr
    0.00126449
    -> 355.0 113.0 f/ pierr
    2.66764e-07
    -> ^D
    %
\end{verbatim}

We can also load programs from files into Interactive \atlast .
Suppose we want to investigate the behaviour of Leibniz' famous
1673 series that converges (achingly slowly) to $\pi$.  The series is:

\[ \frac{\pi}{4} = 1-\frac{1}{3}+\frac{1}{5}-\frac{1}{7}+\frac{1}{9}-
    \cdots \]

We can create a file, using the text editor of our choice, containing
the following:

\begin{verbatim}

\   Series approximations of Pi

\   Leibniz: pi/4 = 1 - 1/3 + 1/5 - 1/7 ...

: leibniz      ( n -- fpi )
    1.0 1.0
    4 pick 1 do
        2.0 f+  \ denom += 2
        2dup
        i 1 and if
            fnegate
        then
        1.0 2swap f/
        2rot f+
        2swap
    loop
    2drop
    rot drop
    4.0 f*
;

\   Reference value of Pi

1.0 atan 4.0 f* 2constant pi

\ Calculate and print error

: pierr
    pi f- fabs f. cr
;
\end{verbatim}

If this seems like gibberish, don't worry!  Remember the first time
you looked at a Lisp or C program.  If you want to decode some of the
structure of this program before learning the language, refer to the
definitions of \atlast\ primitives at the back of this manual, remember
that \atlast\ is a reverse Polish stack language, and note that
``\verb+\+'' is a comment delimiter that causes the rest of the line
to be ignored and that ``{\tt (}'' is a comment delimiter that ignores
all text until the next ``{\tt )}''.

If this file is saved as {\tt leibniz.atl},
we can load the program into Interactive \atlast\ with the command:

{\tt atlast -ileibniz}

\atlast\ will compile the program in the file, report any errors, and
if no errors are found enter the interactive interpretation mode.
The definition of {\tt leibniz} performs the number of iterations
specified by the number on the top of the stack and leaves the
resulting series approximation to $\pi$ on the top of the stack.

We can play with this definition as follows:

\begin{verbatim}
    % atlast -ileibniz
    10 leibniz f.
    3.04184 -> 100 leibniz f.
    3.13159 -> 1000 leibniz f.
    3.14059 -> 10000 leibniz f.
    3.14149 ->   
\end{verbatim}

Well, we can see it's converging, but not very fast.  Since we can
define new compiled words on the fly, let's improvise a definition
that will print the value and its error for increments of 10000
iterations, then run that program.  Continuing our session above:

\begin{verbatim}
    -> : itest 0 do i 1+ 10000 * dup .
    :> leibniz 2dup f. pierr  loop ;
    -> 5 itest
    10000 3.14149 0.0001 
    20000 3.14154 5e-05 
    30000 3.14156 3.33333e-05 
    40000 3.14157 2.5e-05 
    50000 3.14157 2e-05 
    -> ^D
    %
\end{verbatim}

As you can see (even if you don't understand), we've mixed compiled
code, interpreted code, and on-the fly definition of new compiled
functions in a seamless manner.

You can also run an \atlast\ program in batch mode simply by specifying
its name on the {\tt atlast} command line.  If, for example, you added
the lines:

\begin{verbatim}
\   Run iteration vs. error report

: itest
    0 do 
        i 1+ 10000 * dup . leibniz
        2dup f. pierr
    loop
;

10 itest
\end{verbatim}

to the end of the {\tt leibniz.atl} file, creating a new file called
{\tt leibbat.atl}, you could run the program in batch mode as follows:

\begin{verbatim}
% atlast leibbat 
10000 3.14149 0.0001 
20000 3.14154 5e-05 
30000 3.14156 3.33333e-05 
40000 3.14157 2.5e-05 
50000 3.14157 2e-05 
60000 3.14158 1.66667e-05 
70000 3.14158 1.42857e-05 
80000 3.14158 1.25e-05 
90000 3.14158 1.11111e-05 
100000 3.14158 1e-05 
%
\end{verbatim}

(By the way, as is apparent, this is clearly no way to compute $\pi$!
Try this, instead, if you're serious about pumping $\pi$.)

\begin{verbatim}
\   Tamura-Kanada fast Pi algorithm

2variable a
2variable b
2variable c
2variable y

: tamura-kanada ( n -- fpi )
    1.0 a 2!
    1.0 2.0 sqrt f/ b 2!
    0.25 c 2!
    1.0
    rot 1 do
        a 2@ 2dup y 2!
        b 2@ f+ 2.0 f/ a 2!
        b 2@ y 2@ f* sqrt b 2!
        c 2@ 2over a 2@ y 2@ f-
        2dup f* f* f- c 2! 2.0 f*
    loop
    2drop
    a 2@ b 2@ f+ 2dup f* 4.0 c 2@ f* f/
;
\end{verbatim}

\newcommand{\au}[1]{{\tt atl\underline{ }#1}}
\newcommand{\AU}[1]{{\tt ATL\underline{ }#1}}

\section{Debugging}

As befits an interactive language, \atlast\ provides debugging support.
You can trace through the execution of a program word by word by
enabling the {\tt TRACE} facility.  To turn tracing on, enter the
sequence:

{\tt 1 trace}

If you've loaded a definition of the factorial function as follows:

\begin{verbatim}
: factorial
        dup 0= if
           drop 1
        else
           dup 1- factorial *
        then
;
\end{verbatim}

and execute it under trace, you'll see output as
follows:

\begin{verbatim}
% atlast -ifact
-> 1 trace
-> 3 factorial .

Trace: FACTORIAL 
Trace: DUP 
Trace: 0= 
Trace: ?BRANCH 
Trace: DUP 
Trace: 1- 
Trace: FACTORIAL 
Trace: DUP 
Trace: 0= 
Trace: ?BRANCH 
Trace: DUP 
Trace: 1- 
Trace: FACTORIAL 
Trace: DUP 
Trace: 0= 
Trace: ?BRANCH 
Trace: DUP 
Trace: 1- 
Trace: FACTORIAL 
Trace: DUP 
Trace: 0= 
Trace: ?BRANCH 
Trace: DROP 
Trace: (LIT) 1 
Trace: BRANCH 
Trace: EXIT 
Trace: * 
Trace: EXIT 
Trace: * 
Trace: EXIT 
Trace: * 
Trace: EXIT 
Trace: . 6 -> ^D
%
\end{verbatim}

You can turn off tracing with ``{\tt 0 trace}''.

When an error occurs, a walkback is normally printed that lists the
active words starting with the one in which the error occurred,
proceeding through levels of nesting to the outermost, interpretive
level.  If the {\tt WALKBACK} package is configured (see
page~\pageref{`wback'}), the walkback is printed by default.  You can
disable it with ``{\tt 0 walkback}''.  Here is a sample error
walkback report:

\begin{verbatim}
% atlast -ileibniz
-> leibniz
Stack underflow.
Walkback:
   ROT
   LEIBNIZ
->
\end{verbatim}

\section{Integrating \atlast}

Unlike most languages, \atlast\ is not structured as a main program; it
is a subroutine.  You can invoke it when and where you like within
your application, providing as much or as little programmability as is
appropriate.  Before we get into the details of the interface between
an application and \atlast , it's worth showing, by example, just how
simple a program can be that accesses all the facilities of
\atlast\ mentioned so far.  The following main program, linked with the
\atlast\ object module, constitutes a fully-functional interactive
\atlast\ interpreter.  It lacks the refinements of Interactive \atlast\
such as console break processing, batch mode, loading definition
files, prompting with compilation state, and the like, but any program
that Interactive \atlast\ will run can be run by this program, if
submitted to it by input redirection.

\begin{verbatim}
#include <stdio.h>
#include "atlast.h"
int main()
{
    char t[132];
    atl_init();
    while (printf("-> "),
           fgets(t, 132, stdin) != NULL)
        atl_eval(t);
    return 0;
}
\end{verbatim}

\subsection{Configuring {\tt atlast.c}}

\label{`gconf'}
The first step in integrating \atlast\ is building a suitable version
of {\tt atlast.c} that can be linked with your application.  In order
to do this, you must choose the modes with which you wish \atlast\
built.  These modes are normally specified by compile-time
definitions supplied on the C compiler call line.  Unless you request
individual configuration of \atlast\ subpackages, a fully functional
version of \atlast\ will be built.  In that case, you need only be
concerned with the settings of the following compile-time variables.

\begin{description}
\item[{\tt ALIGNMENT}.] If double precision floating point numbers
                    must be aligned on 8 byte boundaries in memory,
                    define {\tt ALIGNMENT}.  If not defined, \atlast\
                    assumes that 4 byte alignment is adequate for
                    these numbers.  (Conditional code in {\tt
                    atldef.h} attempts to define {\tt ALIGNMENT} on
                    processors which require it, but its tests may
                    omit your machine.)  \\
\item[{\tt COPYRIGHT}.]  If you require a statement of the the public
		    domain status of \atlast\ to be embedded into the
		    binary program, define this variable.  Otherwise,
		    leave it undefined and save a few bytes.
\item[{\tt EXPORT}.]  If you are simply invoking \atlast\ as a macro
                    engine and do not require access to its internal
                    data structures, leave {\tt EXPORT} undefined.  If
                    your program adds application-specific
                    primitives to \atlast\ (as most do), define {\tt
                    EXPORT} and include the file {\tt atldef.h} in
                    all modules that require that access.  The stack,
                    return stack, and heap pointers will be made
                    external, names of internal symbols within \atlast\
                    will be redefined to special names beginning with
                    \verb+atl__+ to avoid conflicts with your program,
                    and additional interface code is enabled to
                    provide your primitives full access to the \atlast\
                    runtime environment. \\
\item[{\tt MEMSTAT}.] If you want to enable the runtime memory usage
                    monitor, accessible from the {\tt MEMSTAT}
                    primitive or the \verb+atl_memstat()+ function
                    call, define {\tt MEMSTAT}. \\
\item[{\tt NOMEMCHECK}.] To disable all runtime stack, heap, and
                    pointer checking, define {\tt NOMEMCHECK}\@.  This
                    will yield a dramatic increase in execution speed,
                    but should be enabled only in closed applications
                    after you're sure all the bugs are securely in
                    hiding.  When built with {\tt NOMEMCHECK}, an
                    \atlast\ program is no more secure than a
                    pointer-mad C program.

\item[{\tt READONLYSTRINGS}.] When the {\tt WORDSUSED} package (see
		    page~\pageref{`wordsused'}) is enabled, \atlast\ 
		    keeps track of which primitive and user-defined
		    words are used in a program, allowing you to
		    determine which packages are required and whether
		    your tests have invoked all of the words you have
		    defined. This is done by setting a flag in the
		    word definition which, for built-in primitive
		    words, involves modifying a C constant string.  If
		    your C language implementation does not permit
		    this, define {\tt READONLYSTRINGS}, which will
		    copy the predefined words to a dynamically
		    allocated buffer which may be modified. Note that
		    this is done only if the {\tt WORDSUSED} package
		    is enabled.

\end{description}

When building \atlast\ on MS-DOS or OS/2, you must use a large data
model (32 bit data addresses).  \atlast\ treats all integers as 32
bits and assumes that data pointers are at least that long.
Attempting to build with 16 bit data addresses will cause compile
errors that indicate violation of design assumptions.

\subsection{Initialising: \au{init}}

Before your application makes any other calls to \atlast , you must
call \verb+atl_init+ to initialise its dynamic storage and create
the data structures used to evaluate \atlast\ expressions.

To initialise \atlast\ with the default memory configuration, just
call:

\verb+atl_init();+

The stack, return stack, heap, and initial dictionary are created and
\atlast\ is prepared for execution.  You can adjust the size of the
memory allocated by \atlast\ by setting the following variables
(defined in {\tt atlast.h}) before calling \verb+atl_init+.
\label{`memall'}

\begin{description}
\item[\au{stklen}]          Evaluation (data) stack length.  Expressed
                            as a number of 4 byte stack items.
                            Default 100. \\
\item[\au{rstklen}]         Return stack length.  Expressed as a
                            number of 4 byte return stack pointer
                            items.  Default 100. \\
\item[\au{heaplen}]         Heap length.  Specified as a number of
                            4 byte stack items.  Default 1000. \\
\item[\au{ltempstr}]        Temporary string length.  Gives the
                            length of the buffers used to hold
                            temporary strings entered in interpretive
                            mode and created by certain primitives.
                            Default 256. \\
\item[\au{ntempstr}]        Number of temporary strings.  Specifies
                            the number of temporary strings.
                            Temporary strings are used in rotation; if
                            more than \verb+atl_ntempstr+ are used
                            without storing out the oldest result,
                            it will be overwritten.
                            Default 4.
\end{description}

Applications can allow \atlast\ programs they load to override default
memory allocation specifications with {\em prologue statements}.  See
page~\pageref{`prologue'} for details.  Deeply embedded applications,
such as those programmed into ROMs, may wish to assign the \atlast\
dynamic storage areas to predefined areas of memory instead of
requesting them with {\tt malloc()}.  If the base address pointer of
an area is set nonzero before \verb+atl_init+ is called, the address
specified will be used for that region; no buffer will be allocated.
If you take advantage of this facility, please read the code for
\verb+atl_init()+ in {\tt atlast.c} carefully and make sure the storage
you supply is as long as the various length cells specify.  Note in
particular that the system state word, temporary string buffers, and
heap are consolidated into one contiguous area of memory.

\subsection{Evaluating: \au{eval}}

To evaluate a string containing \atlast\ program text, call:

{\tt {\em stat} = \verb+atl_eval+({\em string});}

where {\em string} is a string containing the text to be evaluated and
{\em stat} is an integer giving the status of the evaluation.
Mnemonics for evaluation status codes are defined in {\tt atlast.h},
and have the following meanings:

\begin{tabbing}
\AU{RSTACKUNDER} \= Attempt to {\tt FORGET} protected \kill
\AU{SNORM} \>          No error \\
\AU{STACKOVER} \>      Stack overflow \\
\AU{STACKUNDER} \>     Stack underflow \\
\AU{RSTACKOVER} \>     Return stack overflow \\
\AU{RSTACKUNDER} \>    Return stack underflow \\
\AU{HEAPOVER} \>       Heap overflow \\
\AU{BADPOINTER} \>     Bad heap pointer \\
\AU{UNDEFINED} \>      Undefined word \\
\AU{FORGETPROT} \>     Attempt to {\tt FORGET} protected \\
                \>        symbol \\
\AU{NOTINDEF} \>       Compiler word outside definition \\
\AU{RUNSTRING} \>      Runaway string \\
\AU{RUNCOMM} \>        Runaway comment in file \\
\AU{BREAK} \>          Asynchronous break signal \\
           \>          received \\
\AU{DIVZERO} \>        Attempt to divide by zero \\
\end{tabbing}

In addition to these status codes, a program that calls
\verb+atl_eval+ may determine the current state of \atlast\ by
examining external variables.  If a multi-line comment awaiting
termination with a ``{\tt )}'' is active, \verb+atl_comment+ will be
nonzero.  If the definition of a word (colon definition) is currently
pending, the variable {\tt state} (accessible only if {\tt EXPORT} is
defined and {\tt atldef.h} is included) will be nonzero.

\subsection{Loading files: \au{load}}

To load an entire file containing \atlast\ program text, call:

{\tt {\em stat} = \verb+atl_load+({\em file});}

where {\em file} is a C file descriptor (type {\tt FILE *})
designating the file, currently open for input and positioned before
the first byte of the \atlast\ program to be loaded.  The
program is read, and {\em stat} is the status resulting from
loading and executing the \atlast\ program in that file.  The status
codes are the same as those given above for the \verb+atl_eval+
function.  The \verb+atl_load+ function reads text files in any of the end
of line conventions recognised by AutoCAD; ASCII files in any of these
formats may be loaded by any implementation of \atlast .  If the host
system requires binary files to be identified at open time, files
containing \atlast\ programs to be loaded with \verb+atl_load+ should
be opened in {\em binary} mode, even though they nominally contain ASCII
text.  Binary mode permits correct interpretation of all the end of
line delimiters accepted by AutoCAD\@.

The \verb+atl_load+ function uses \verb+atl_mark+ to save the runtime
status before loading the file.  If an error occurs, it attempts to
restore the {\em status quo ante} by performing an \verb+atl_unwind+.
If the file loaded included interpretive mode code that modified
preexisting objects on the heap, those changes will not be reversed if
an error occurs whilst loading the file.

\subsection{Marking: \au{mark}}

Applications may wish to undertake a series of \atlast\ operations
which might result in a runtime evaluation error.  In that event, the
application will normally want to undo definitions made by the
program that errored.  To mark one's place before embarking upon a
potentially perilous \atlast\ program, use:

{\tt \verb+atl_statemark+ mk;\\
\verb+atl_mark(&mk);+}

The current position of the stack, return stack, heap, and dictionary
are saved in the \verb+atl_statemark+ structure.  A subsequent
\verb+atl_unwind+ call will roll each of those dynamic storage areas
back to the position at the designated \verb+atl_mark+.

\subsection{Reversing changes: \au{unwind}}

To roll back all changes to the stack, return stack, heap allocation,
and dictionary to the state saved in an \verb+atl_statemark+ object
with \verb+atl_mark+, call:

{\tt \verb+atl_statemark+ mk;\\
\verb+atl_unwind(&mk);+}

The allocation pointers for all the storage areas are reset to their
positions at the time \verb+atl_mark+ was called, but changes to
heap variables made by storing through pointers after the
\verb+atl_mark+ are not reversed.

\subsection{Asynchronous break: \au{break}}

Interactive applications of \atlast\ must allow the user
to escape infinite loops and other accidentally initiated lengthy
computations.  If the system provides a facility for responding to
user interrupt requests, \atlast\ allows execution of programs under
its control to be terminated through the \verb+atl_break+ mechanism.

If {\tt BREAK} is defined at compile time, the \verb+atl_break()+
function and support for asynchronous break is enabled.  When the
application receives an asynchronous break, it should call
\verb+atl_break()+ to notify the currently running \atlast\ program of
the break signal.  If no \atlast\ program is running at the time of the
signal, no harm is done.  The application break routine should always
call \verb+atl_break()+ rather than try to determine whether \atlast\
is active.  If an \atlast\ program was executing at the time of the
break signal, the application that invoked it, whether by
\verb+atl_eval+, \verb+atl_load+, or \verb+atl_exec+, will be notified
of the abnormal termination by the return of the \verb+ATL_BREAK+
status.

The \verb+atl_break+ function simply sets a flag examined by the inner
loop of the \atlast\ evaluator; it does not actually terminate
execution.  Consequently, it may safely be called at any time, even
from hardware interrupt service routines.

\subsection{Showing memory status: \au{memstat}}

In the final stage of optimising an application incorporating \atlast\
for shipment, one may wish to adjust the memory allocation parameters
to eliminate wasted space while providing reasonable margins for user
extensions after shipment.  To set the parameters wisely, one must
know the baseline memory usage of the application.  If {\tt atlast.c}
is built with {\tt MEMSTAT} defined, this can be obtained either by
executing the {\tt MEMSTAT} primitive within the \atlast\ program or by
calling the \verb+atl_memstat+ function at an opportune time within
the application.  In either case, a memory usage report similar to the
following example is written to the standard output stream.

{\small
\begin{verbatim}
             Memory Usage Summary

              Current  Maximum  Items   Percent
Memory Area    usage    used  allocated in use 
 Stack            0        9     100       0
 Return stack     0        4     100       0
 Heap           227      227    1000      22
\end{verbatim}
}

\vspace{1ex}
\centerline{\fbox{\parbox[t]{19pc}{\large {\bf Note:} to use any of the
 following functions, you must compile {\tt atlast.c} and the modules
 that call them with {\tt EXPORT} defined, and you must include the
 header file {\tt atldef.h} in files that call them.}}}

\subsection{Looking up words: \au{lookup}}

Your application can look up words in the \atlast\ dictionary, using
the same search order as the interpreter would, with the
call:

{\tt dictword *{\em dw};\\
char *{\em name};\\
\\
{\em dw} = \verb+atl_lookup+({\em name});}

Since \atlast\ names are matched regardless of whether letters in them
are upper or lower case, the {\em name} may contain any combination of upper
and lower case letters.  If the word is defined, its dictionary entry
is returned.  The {\tt dictword} structure is defined in {\tt
atldef.h}.  If the word is not defined, {\tt NULL} is returned.
There may be multiple nested definitions of a word; if this is the
case, only the most recent definition (the active definition) is
returned.  There is no way, using \verb+atl_lookup+ alone, to locate
hidden definitions.

\subsection{Accessing a word's text: \au{body}}

An \atlast\ word definition consists of several components, including
its name and the C-coded method that implements it.  Of most interest
to applications that intercommunicate with \atlast\ is the {\em body\/}
of the word.  For a variable or constant, this is the storage that
contains the word's value.  To obtain the body address of a
dictionary item returned by \verb+atl_lookup+ or created by
\verb+atl_vardef+ (see below), use \verb+atl_body+.  The call:

{\tt
dictword *{\em dw};\\
stackitem *{\em si};\\
\\
{\em si} = \verb+atl_body+({\em dw});}

places the body address of dictionary item {\em dw} into variable {\em
si}.  If you wish to store a data type into the body of the
\atlast\ word other than the default of {\tt stackitem} (defined as {\tt
long}), cast the pointer to the correct pointer type.  See the
\verb+atl_vardef+ sample below for an example of a floating point
variable being created and initialised using \verb+atl_body+.

\subsection{Defining variables: \au{vardef}}

Shared variables are
a convenient way of intercommunicating between a host application
and \atlast .  By making the application's state
visible to and changeable by the \atlast\ program, the program is given the
information it needs and the power to direct the application.  A
shared variable is an \atlast\ variable defined by the application, the
address of which is known both to \atlast\ (via the dictionary), and to
the application (by a pointer returned when the shared variable is
created).  To create a shared variable, call:

{\tt dictword *{\em var};\\
\\
{\em var} = \verb+atl_vardef+({\em name}, {\em size});}

where {\em name} is a character pointer giving the name of the
variable to be created and {\em size} is an integer specifying its
size in bytes.  Note that to create a normal \atlast\ integer variable
{\em size} should be 4; for a floating point variable, {\em size}
should be 8 bytes.  Storage for the variable is reserved on the
\atlast\ heap.  If insufficient heap space is available to create the
variable {\tt NULL} is returned.  Otherwise, the address of the
variable's dictionary entry is returned.  {\bf Beware:} the dictionary
entry {\em is not} the storage address of the variable's value.
To obtain that address, call \verb+atl_body+, described above.

For example, we can create a floating point variable containing a
crummy approximation of $\pi$ with the sequence:

\begin{verbatim}
dictword *pi;

pi = atl_vardef("Pi", sizeof(double));
if (pi == NULL) {
    printf("Can't atl_vardef PI.\n");
} else {
    *((double *) atl_body(pi)) =
        3.141596235;
}
\end{verbatim}

We could then print the value with an \atlast\ program run under that
application with:

{\tt pi 2@ f. }

\subsection{Executing words: \au{exec}}

If you've obtained the dictionary address of an \atlast\ word
definition, your application can execute it with the sequence:

{\tt dictword *{\em dw};\\
int {\em stat};
\\
{\em stat} = \verb+atl_exec+({\em dw});}

The status codes returned in {\em stat} are identical to those
returned by \verb+atl_eval+.  The distinction between \verb+atl_eval+
and \verb+atl_exec+ is subtle, but important---it can make a big
difference in the performance of your application.  If you know the
name of an \atlast\ word, you can execute it either by passing a string
containing its name to \verb+atl_eval+ or by saving its dictionary
address in a variable and executing the word directly from the
dictionary address with \verb+atl_exec+.  The results of these two
operations are identical, but when you pass a string to
\verb+atl_eval+, \atlast\ is forced to scan the string, parse its
contents into the token denoting the word, look that word up in the
dictionary, and only then execute the word.  You can bypass all these
nonproductive and time consuming preliminaries if you know the word's
dictionary address and use \verb+atl_exec+.

Creative use of \verb+atl_lookup+ and \verb+atl_exec+ provide one of
the most powerful ways for \atlast\ to enrich an application.  If you
create an application to perform a relatively well-defined task, you
can, before entering its main processing loop,
inquire with \verb+atl_lookup+ whether the user has defined a series
of words specified by the application.  If so, their dictionary
addresses are saved in pointers in the application code.  Then, as the
application executes, at each step where the user might want to
interpose his own processing or replace the application's default
processing with his own method, the application merely tests whether
the word associated with that step has been defined in the \atlast\
program and, if so, runs it with \verb+atl_exec+.  If the
default processing that would otherwise occur is made available
as an \atlast\ primitive with \verb+atl_primdef+ (see below), it is
extremely easy for the \atlast\ program to examine the data at the
point it has been ``hooked,'' perform any special processing
it wishes, or inherit the default processing simply by running
the primitive that does it.  If the user has not requested special
processing, the cost to the application to provide that opportunity
is one pointer comparison against {\tt NULL}\@.
Compared with the benefits of open architecture, this is a small price
indeed.

You can pass arguments to the definition you're invoking with
\verb+atl_exec+ either by storing them in shared variables created
with \verb+atl_vardef+ or, usually the best approach, pushing them on
the stack before executing the definition.  See the discussion of
\verb+atl_primdef+ below for information on access to the stack
from C\@.

\subsection{Defining primitives: \au{primdef}}

Most of the power of \atlast\ derives from the ease with which C coded
primitives can be added to the language.  Once integrated, they may be
used in conjunction with the looping, conditional execution, and other
facilities already present.  \atlast\ has been
deliberately designed to make the addition of primitives simple and
safe: nothing like the peril-filled nightmare of adding a function to
AutoLISP.  Still, to extend any language you need to learn your way
around its memory architecture and control structure.  So, listen up,
walk through the examples, and before long you'll be adding primitives
like a pro.

An \atlast\ primitive is a C function.  When the primitive is executed,
that function is called and may do whatever it likes.  A primitive can
be as simple as one that discards the top item on the stack, or as
complex as one that prepares a ray-traced bitmap from a three
dimensional geometric model.  Most primitives communicate with one
another via the {\em stack}.  Some primitives also access variables
stored on the {\em heap}.  Finally, a very few primitives manipulate
data stored on the {\em return stack}, which \atlast\ uses to track the
nesting of execution.  A user-defined primitive will rarely need to
access the return stack.  Definitions in {\tt atldef.h} simplify
access to each of these areas of memory.  Let's look at them one by
one.

\subsubsection{Accessing the stack}

The stack pointer variable is called {\tt stk}, and always points to
the next available {\tt long} stack item.  Primitives rarely reference {\tt
stk} directly, since it is usually far more convenient to use
definitions that hide the complexity of indexing the stack.  The
following tools are provided for access to the stack.

\begin{description}
\item[{\tt Sl({\em n})}]        Before you access any items on the
                                stack, you must check that the stack
                                actually contains at least as many 
                                items as you'll be using.  If not,
                                a {\em stack underflow} must be
                                reported.  At the start of your
                                primitive, simply use the statement
                                ``{\tt Sl({\em n});}'', where {\em n}
                                is the number of stack items you'll be
                                referencing.  If you use the topmost
                                two stack items, {\tt S0} and {\tt
                                S1}, you'd use {\tt Sl(2);}.  It's
                                important that you use the definition
                                rather than check the stack limit
                                directly; if you later build your
                                application with stack checking off,
                                the {\tt Sl()} statement will generate
                                no code, automatically configuring
                                your primitive for maximum speed.
\item[{\tt So({\em n})}]        Before you push any new items onto the
                                stack, you must check that the stack
                                will not overflow the area allocated
                                to it when those items are added.
                                If it would,
                                a {\em stack overflow} must be
                                reported.  At the start of your
                                primitive, simply use the statement
                                ``{\tt So({\em n});}'', where {\em n}
                                is the number of new stack items you'll be
                                pushing.  If you are adding one new
                                integer item to the stack, use ``{\tt
                                So(1);}''.  It's
                                important that you use the definition
                                rather than check the stack limit
                                directly; if you later build your
                                application with stack checking off,
                                the {\tt So()} statement will generate
                                no code, automatically configuring
                                your primitive for maximum speed.
\item[{\tt S0}--{\tt S5}]       The definitions {\tt S0}, {\tt
                                S1},\ldots {\tt S5} provide direct
                                access to the top 6 integer stack
                                items.  {\tt S0} is the top item on
                                the stack, {\tt S1} is the next item,
                                and so on.  These definitions may be
                                used on either the left or right
                                side of an assignment.
\item[{\tt Pop}]                Used as a statement, ``{\tt Pop;}'',
                                discards the topmost item from the
                                stack.
\item[{\tt Pop2}]               Used as a statement, ``{\tt Pop2;}'',
                                discards the topmost two items from the
                                stack.
\item[{\tt Npop({\em n})}]      Discards the top {\em n} items from
                                the stack.
\item[{\tt Push}]               Used on the left side of an
                                assignment, stores the value on the
                                right side into the next free stack
                                item and increments the stack pointer.
\item[{\tt Realsize}]           For primitives that use floating point
                                numbers, {\tt Realsize} gives the
                                number of stack items occupied by one
                                floating point number.  A primitive
                                that expects two floating point
                                arguments on the stack and will leave
                                them there, adding one new floating
                                point result would begin ``{\tt Sl(2 *
                                Realsize); So(Realsize);}''.
\item[{\tt REAL0}--{\tt REAL2}] These definitions provide read access
                                to the topmost three floating point
                                numbers on the stack.  The stack cells
                                are automatically cast to type {\tt
                                double}.  It is {\em essential} that
                                you access floating point values this
                                way---some computers require that {\tt
                                double}s be aligned on 8 byte
                                boundaries, and the {\tt REAL{\em n}}
                                definitions automatically align the
                                variable if the machine requires it.
\item[{\tt SREAL0({\em f})}, {\tt SREAL1({\em f})}] These definitions,
                                used as functions, store their
                                floating point arguments into the
                                topmost ({\tt SREAL0}) and next ({\tt
                                SREAL1}) floating point items on the
                                stack.  Because of the possible need
                                to compensate for machine alignment
                                restrictions, the {\tt REAL{\em n}}
                                definitions cannot be used on the left
                                side of an assignment; use these
                                functions instead.
\item[{\tt Realpop}]            Pops the topmost floating point value
                                from the stack.  Equivalent to {\tt
                                Npop(Realsize)}.
\item[{\tt Realpop2}]           Pops the two topmost floating point
                                values
                                from the stack.  Equivalent to {\tt
                                Npop(2 * Realsize)}.
\end{description}

{\em He said this was easy!}  Please bear with me---all of this is
far simpler (and more compact) to use than it is to explain.  If you
can't stand it, skip ahead to the sample primitive definitions and see
for yourself.  O.K., welcome back.  Probably 95\% of all the
primitives you'll add to \atlast\ will confine themselves to accessing
the stack.  Heap and return stack access is far less frequent (and may
indicate poor design).  In any case, if you need to do it, here's
how.

\subsubsection{Accessing the heap}

The {\em heap} is a pool of memory used to allocate static
objects.  Most heap is allocated by \atlast\ {\em defining words}, such
as {\tt VARIABLE}, {\tt CONSTANT}, and the {\tt :} used to define new
executable words, themselves stored on the heap.  The ability to
create defining words for new data types directly in \atlast\ is one of
its most powerful features and reduces the need to manipulate the heap
from user primitives.  The heap is accessed through a set of
definitions similar to those used for the stack.  The heap pointer
itself is named {\tt hptr}, but will rarely be referenced explicitly.

\begin{description}
\item[{\tt Ho({\em n})}]        Before you store any new data on the
                                heap, you must verify that doing so
                                would not cause the heap to grow past
                                its assigned maximum size.  This event
                                is called a {\em heap overflow}, and
                                the {\tt Ho({\em n})} function checks
                                for it and terminates execution should
                                overflow occur.  The number {\em n} is
                                the amount of heap you propose to
                                allocate, {\em in terms of stack
                                items}, each of four bytes.  If you
                                wish to allocate a number expressed in
                                bytes, you must round it up to the
                                next larger multiple of four.  A
                                portable way to do this is to use the
                                expression: {\tt (({\em x} +
                                (sizeof(stackitem) - 1)) /
                                sizeof(stackitem))} where {\em x} is
                                the number of bytes of heap you
                                require.  If you configure stack and
                                heap checking off for maximum
                                performance, {\tt Ho({\em n})} generates
                                no code.
\item[{\tt Hpc({\em ptr})}]     Heap storage is normally accessed via
                                pointers passed on the stack.  Since
                                the stack contains many other types of
                                data, accidentally using a non-pointer
                                as a heap address could be
                                catastrophic.  Before using any value
                                as a pointer to the heap, call {\tt
                                Hpc({\em ptr})} where {\em ptr} is the
                                pointer.  If the pointer is not within
                                the heap, a {\em bad pointer} error
                                will be reported and execution
                                terminated.  If you configure stack
                                and heap checking off, {\tt Hpc({\em
                                ptr})} generates no code.
\item[{\tt Hstore}]             Used on the left of an assignment,
                                stores the {\tt long} value on the
                                right side into the next available
                                heap cell and advances the heap
                                allocation pointer.
\end{description}

\subsubsection{Accessing the return stack}

The return stack remembers the point at which one definition
invoked another, tracks loop control indices, and stores other items
internal to the evaluator.  Messing with the return stack is generally
a very bad idea.  This information is presented not so much to
encourage you to use the return stack as for completeness and to
document the code within {\tt atlast.c} that maintains it.
The stack pointer variable is called {\tt rstk}, and always points to
the next available return stack item.  Return stack items have a type of
{\tt **dictword} (got that?), which is also {\tt typedef}ed to {\tt
rstackitem}.

Primitives rarely reference {\tt
rstk} directly, since it is usually far more convenient to use
definitions that hide the complexity of indexing the return stack.  The
following tools provide access to the return stack.

\begin{description}
\item[{\tt Rsl({\em n})}]       Before you access any items on the
                                return stack, you must check that the
                                return stack actually contains at
                                least as many items as you'll be
                                using.  Otherwise, a {\em return stack
                                underflow} must be reported.  At the
                                start of your primitive, simply use
                                the statement ``{\tt Rsl({\em n});}'',
                                where {\em n} is the number of return
                                stack items you'll be referencing.  If
                                you use the topmost two items,
                                {\tt R0} and {\tt R1}, you'd use {\tt
                                Rsl(2);}.  It's important that you use
                                the definition rather than check the
                                return stack limit directly; if you
                                later build your application with
                                stack checking off, the {\tt Rsl()}
                                statement will generate no code,
                                automatically configuring your
                                primitive for maximum speed.
\item[{\tt Rso({\em n})}]       Before you push any new items onto the
                                return stack, you must check that the
                                return stack will not overflow the
                                area allocated to it when those items
                                are added.  If it would, a {\em return
                                stack overflow} must be reported.  At
                                the start of your primitive, simply
                                use the statement ``{\tt Rso({\em
                                n});}'', where {\em n} is the number
                                of new return stack items you'll be
                                pushing.  If you are adding one new
                                item to the return stack, use ``{\tt
                                Rso(1);}''.  It's important that you
                                use the definition rather than check
                                the return stack limit directly; if you
                                later build your application with
                                stack checking off, the {\tt Rso()}
                                statement will generate no code,
                                automatically configuring your
                                primitive for maximum speed.
\item[{\tt R0}--{\tt R2}]       The definitions {\tt R0}, {\tt
                                R1}, and {\tt R2} provide direct
                                access to the top three return stack
                                items.  {\tt R0} is the top item on
                                the return stack, {\tt R1} is the next
                                item, and {\tt R2} is the third item.
                                These definitions may be used on
                                either the left or the right side of
                                an assignment.
\item[{\tt Rpop}]               Used as a statement, ``{\tt Rpop;}'',
                                discards the topmost item from the
                                return stack.
\item[{\tt Rpush}]              Used on the left side of an
                                assignment, stores the value on the
                                right side into the next free return
                                stack item and increments the return
                                stack pointer.
\end{description}

\subsubsection{Coding primitive functions}

Each primitive word you define is implemented by a C function declared
as ``{\tt static void}''.  The header file {\tt atldef.h} defines
``{\tt prim}'' as this type to more explicitly identify primitive
implementing functions.

As an example of a simple primitive, let's add the ability to obtain
the date and time in Unix format and to extract the hours, minutes,
and seconds from the Unix date word.  We'll add two new primitive
functions to \atlast : {\tt TIME}, which leaves the number of seconds
since midnight on January 1, 1970 on the top of the stack, and {\tt
HHMMSS} which, given the value returned by {\tt TIME},
leaves the hours, minutes, and seconds represented by that time in the
three top stack locations, with the seconds at the top.

Here is the C function that implements the {\tt TIME} primitive word:

\begin{verbatim}
prim ptime()
{
    So(1);
    Push = time(NULL);
}
\end{verbatim}

Since we're placing one new word on the stack, we call {\tt So(1)} to
check for stack overflow.  That accomplished, we simply use {\tt Push}
on the left side of the assignment to store the {\tt long} time word
returned by the Unix-compatible {\tt time()} function (which is
supported by most non-Unix C libraries, as well).

The function for our {\tt HHMMSS} primitive is more complicated, but
not much.  It uses the Unix-compatible {\tt localtime()} function
which, passed a pointer to a word containing a time in the format
returned by {\tt time()}, returns a pointer to an internal static
structure with fields that give the day, month, year, hour, minute,
second, etc.\ represented by that time.  The primitive definition is:

\begin{verbatim}
prim phhmmss()
{
    struct tm *lt;

    Sl(1);
    So(2);
    lt = localtime(&S0);
    S0 = lt->tm_hour;
    Push = lt->tm_min;
    Push = lt->tm_sec;
}
\end{verbatim}

This primitive expects one argument (the time word) on the stack, so
it begins with {\tt Sl(1)} to verify that it is present.  It will
replace that value with the hours and add two new items to the stack
for the minutes and seconds, so it next uses {\tt So(2)} to ensure
those additions won't cause the stack to overflow.  Now it can get
down to business.  It calls {\tt localtime()}, passing the address of
the first stack item (the time word), then stores the hours back into
that word and uses {\tt Push} twice to add the minutes and seconds.

Once the primitive functions are coded, the primitives are actually
added to \atlast\ by listing them in a primitive definition table and
registering that table with \atlast\ by calling the \verb+atl_primdef+
function.  The primitive definition table for our two new primitives
is as follows:

\begin{verbatim}
static struct primfcn timep[] = {
    {"0TIME",   ptime},
    {"0HHMMSS", phhmmss},
    {NULL,      (codeptr) 0}
};
\end{verbatim}

The {\tt primfcn} structure is declared in {\tt atldef.h}.  You may
list as many primitives in the table as you wish.  The end of the
table is marked by an entry with {\tt NULL} instead of a primitive
name.  For each primitive you define, make an entry with two components:
the first a string with the first character ``{\tt 0}'' if the
primitive is a normal word and ``{\tt 1}'' if it is a compile-time
immediate word, the balance of which is the name of the
primitive with all letters upper case.  The second component
is the name of the function that implements the primitive.
The primitives in the table are defined by calling
\verb+atl_primdef+, passing the address of the table as follows:

\verb+atl_primdef(primt);+

(Subtle note for MS-DOS users: to save memory, \atlast\ uses the
actual static strings you declare in the primitive table as part of
the dictionary entries it creates.
Since the
\atlast\ dictionary will contain pointers to these compiled-in strings,
you must not place the data for the primitive table in an overlay
which might be swapped out when \atlast\ later attempts to search the
dictionary.  If your program does not overlay its data segment, you
need not worry about this.)

You can call \verb+atl_primdef+ any time after you've called
\verb+atl_init+, and you can call it as many times as you like with
different {\tt primfcn} tables.  If a name in a {\tt primfcn} table
duplicates the name of a built-in \atlast\ primitive or a primitive
defined by an previous call on \verb+atl_primdef+, the earlier
definition will be hidden and inaccessible.

With these new primitives installed, we can now try them out interactively from
\atlast .

\begin{verbatim}
% atlast
-> time .
634539503 -> time .
634539505 -> time .
634539508 -> time .s
Stack: 634539512 -> hhmmss
-> .s
Stack: 20 58 32 -> clear time hhmmss .s
Stack: 20 58 44 -> clear
-> time hhmmss .s
Stack: 20 58 52 -> ^D
%
\end{verbatim}

Everything seems to be behaving as we intended.  Our new primitives
work!

Finally, let's look at a more complicated primitive, one involving
floating point.  Turning again to the Leibniz series for $\pi$, here
is the C language definition of a primitive function to evaluate
it.  The function is compatible with the one we previously implemented
in \atlast : it expects the number of terms on the top of the stack and
returns the approximation of $\pi$ as a floating point value in the two
top stack items.

\begin{verbatim}
prim pleibniz()
{
    long nterms;
    double sum = 0.0,
           numer = 1.0,
           denom = 1.0;

    Sl(1);
    nterms = S0;
    Pop;

    So(Realsize);
    Push = 0;
    Push = 0;
    while (nterms-- > 0) {
        sum += numer / denom;
        numer = -numer;
        denom += 2.0;
    }
    SREAL0(sum * 4.0);
}
\end{verbatim}

This function begins by verifying with {\tt Sl(1)} that its term count
argument is present on the stack.  It loads that argument, referenced
as {\tt S0}, and saves it in the loop count, {\tt nterms}.  The
iteration count is then discarded from the stack with {\tt Pop}.
Next, {\tt So(Realsize)} verifies that the stack will not overflow
when the real result is pushed (recall that {\tt Realsize} is the
number of stack items per floating point result---this is always two,
but using the definition makes for more readable code).  We then
immediately count on {\tt Realsize} being two as we use two {\tt Push}
operations to allocate the stack space for the result and clear it to
zero.  That done, the function falls into the loop that sums the
requested number of terms of the series.  Finally, {\tt SREAL0()} is
used to store the result into the top floating point value on the
stack: the one we created with the two {\tt Push}es.

This primitive is declared and registered with \atlast\ with the
sequence:

\begin{verbatim}
static struct primfcn pip[] = {
    {"0LEIBNIZ", pleibniz},
    {NULL, (codeptr) 0}
};
atl_primdef(pip);
\end{verbatim}

With a C coded primitive implementation, we can explore the outer
reaches of this awful series.  For example, here it's used to print
the error after the first half million terms.

\begin{verbatim}
% atlast
-> 2variable pi
-> 1.0 atan 4.0 f* pi 2!
-> pi 2@ f. c 
3.14159
-> 500000 leibniz pi 2@ f- f. cr
-2e-06
-> ^D
%
\end{verbatim}

As you can see from the brevity and straightforwardness of these
sample
primitives, there's nothing complicated or difficult about adding a
primitive to \atlast .  The overhead in executing a primitive
function from \atlast\ rather than calling it from a C program is a
matter of a few instructions.  If you need guidance in implementing
primitives that interact with \atlast\ in more intricate ways, the best
source of information is the source code of {\tt atlast.c}; find
a standard primitive with arguments and results similar to the one
you're planning to add, and look up its implementing function.  That
should abate any confusion about the fine points of stack and heap
manipulation.

\section{Package configuration}

In addition to the global configuration parameters described on
page~\pageref{`gconf'}, you can choose precisely which components of
\atlast\ are included when building a version for your application by
creating a custom configuration file named {\tt custom.h}, then
compiling {\tt atlast.c} with the {\tt -DCUSTOM} compiler flag.  A
custom configuration file has the following format:

\begin{raggedright}
\verb+#define INDIVIDUALLY+\\
\verb+#define +{\em Package$_1$}\\
\verb+#define +{\em Package$_2$}\\
\hspace*{8em} $\vdots$ \\
\verb+#define +{\em Package$_n$}
\end{raggedright}

The {\em Package$_n$} definitions select which \atlast\ subpackages you
wish included in your application.  The individual subpackages are
described in the following paragraphs.  The {\tt WORDSUSED} and {\tt
WORDSUNUSED} primitives, available as part of the {\tt WORDSUSED}
package, let you determine which primitives are used within an \atlast\
program and, consequently, which packages are required to execute it.

\paragraph{The {\tt ARRAY} package.}
Provides declaration of $n$ dimensional arrays of arbitrary data types
and runtime subscript calculation for such arrays.  Primitives: {\tt
ARRAY}.

\paragraph{The {\tt BREAK} package.}
Enables asynchronous break processing via the \verb+atl_break+
function.  Disabling this package saves an insignificant amount of
memory but increases execution speed by about 10\%.  Primitives: none.

\paragraph{The {\tt COMPILERW} package.}
Enables primitives used to define new compiler words.  Primitives:
{\tt [COMPILE]}, {\tt LITERAL}, {\tt COMPILE}, {\tt <MARK}, {\tt
<RESOLVE}, {\tt >MARK}, {\tt >RESOLVE}.

\paragraph{The {\tt CONIO} package.}
Enables primitives that display interactive output.  These primitives
may be disabled in applications that provide no interaction with the
user.  Primitives: {\tt .}, {\tt ?}, {\tt CR}, {\tt .S}, {\tt ."},
{\tt .(}, {\tt TYPE}, {\tt WORDS}.

\paragraph{The {\tt DEFFIELDS} package.}
Enables low level primitives used to manipulate dictionary items.
These primitives are rarely used except in very ambitious
language extensions coded in \atlast .  Primitives: {\tt FIND}, {\tt
>NAME}, {\tt >LINK}, {\tt BODY>}, {\tt NAME>}, {\tt LINK>}, {\tt
N>LINK}, {\tt L>NAME}, {\tt NAME>S!}, {\tt S>NAME!}.

\paragraph{The {\tt DOUBLE} package.}
Enables double word operations.  These operations can be used with any
stack data, but are heavily used in floating point code, since
floating point numbers occupy pairs of stack items.  Primitives: {\tt
2DUP}, {\tt 2DROP}, {\tt 2SWAP}, {\tt 2OVER}, {\tt 2ROT}, {\tt
2VARIABLE}, {\tt 2CONSTANT}, {\tt 2!}, {\tt 2@}.

\paragraph{The {\tt FILEIO} package.}
Enables the C language-like file primitives.  If your application does
not require access to files, this package may be disabled.
Primitives: {\tt FILE}, {\tt FOPEN}, {\tt FCLOSE}, {\tt FDELETE}, {\tt
FGETS}, {\tt FPUTS}, {\tt FREAD}, {\tt FWRITE}, {\tt FGETC}, {\tt
FPUTC}, {\tt FTELL}, {\tt FSEEK}, {\tt FLOAD}\@.  In addition, {\tt
FILE} variables {\tt STDIN}, {\tt STDOUT}, and {\tt STDERR} are
defined, automatically bound to the Unix I/O streams with the same
names.

\paragraph{The {\tt MATH} package.}
Enables the mathematical functions.  {\tt MATH} can be enabled only if
{\tt REAL} is also enabled.  Primitives: {\tt ACOS}, {\tt ASIN}, {\tt
ATAN}, {\tt ATAN2}, {\tt COS}, {\tt EXP}, {\tt LOG}, {\tt POW}, {\tt
SIN}, {\tt SQRT}, {\tt TAN}.

\paragraph{The {\tt MEMMESSAGE} package.}
Controls whether messages are printed when runtime errors
(such as stack overflow and underflow, bad pointers, etc.)\ occur.
Disabling these messages doesn't save time or significant memory: it's
intended for deeply embedded applications where returning the error
status to the caller of \verb+atl_eval+ or \verb+atl_exec+ is all the
error notification that is appropriate.  Primitives: none.

\paragraph{The {\tt PROLOGUE} package.}
\label{`prologue'}
The amount of memory allocated to the stack, return stack, heap, and
temporary string buffers can be controlled by setting the external
variables governing those areas as described on
page~\pageref{`memall'}.  You can allow the \atlast\ program text to
override the default settings you make by enabling the {\tt PROLOGUE}
package.  If this package is enabled, special statements of the form:

\verb+\ *+{\em area} {\em size}

are recognised by the evaluator when encountered before the first line
containing executable \atlast\ text.  To permit processing of the
prologue, {\em do not} explicitly call \verb+atl_init+; it will be
called automatically by \verb+atl_eval+ after the prologue is
processed.  The following {\em area} specifications are recognised in
the prologue:

\begin{description}
\item[{\tt STACK}]      Specifies the stack size in terms of {\tt
                        long} stack items.
\item[{\tt RSTACK}]     Specifies the return stack size in items.
\item[{\tt HEAP}]       Specifies the heap size as a number of {\tt
                        long} stack items.
\item[{\tt TEMPSTRL}]   Specifies the length of each temporary string
                        buffer in characters.
\item[{\tt TEMPSTRN}]   Specifies the number of temporary string
                        buffers.
\end{description}

\paragraph{The {\tt REAL} package.}
Enables floating point operations.  If you enable the {\tt REAL}
package, you should also enable the {\tt DOUBLE} package; without it
you won't be able to accomplish much.  Primitives: {\tt (FLIT)}, {\tt
F+}, {\tt F-}, {\tt F*}, {\tt F/}, {\tt FMIN}, {\tt FMAX}, {\tt
FNEGATE}, {\tt FABS}, {\tt F=}, {\tt F<>}, {\tt F>}, {\tt F<}, {\tt
F>=}, {\tt F<=}, {\tt F.}, {\tt FLOAT}, {\tt FIX}.

\paragraph{The {\tt SHORTCUTA} package.}
Enables shortcut integer arithmetic operations.  Primitives: {\tt 1+},
{\tt 2+}, {\tt 1-}, {\tt 2-}, {\tt 2*}, {\tt 2/}.

\paragraph{The {\tt SHORTCUTC} package.}
Enables shortcut integer comparison operations.  Primitives: {\tt 0=},
{\tt 0<>}, {\tt 0<}, {\tt 0>}.

\paragraph{The {\tt STRING} package.}
Enables string operations.  Primitives: {\tt (STRLIT)}, {\tt
STRING}, {\tt STRCPY}, {\tt S!}, {\tt STRCAT}, {\tt S+}, {\tt STRLEN}, 
{\tt STRCMP}, {\tt STRCHAR}, {\tt SUBSTR}, {\tt COMPARE}, {\tt STRFORM}, {\tt
STRINT}, {\tt STRREAL}\@.  If the {\tt REAL} package is also enabled,
the {\tt FSTRFORM} primitive is available, as well.

\paragraph{The {\tt SYSTEM} package.}
Enables submission of commands in strings to the operating system for
execution.  This package may be enabled only if the implementation of
C used to build \atlast\ provides the {\tt system()} function.
Primitives: {\tt SYSTEM}.

\paragraph{The {\tt TRACE} package.}
Enables runtime word execution trace.  Primitives: {\tt TRACE}.

\paragraph{The {\tt WALKBACK} package.}
\label{`wback'}
Enables the walkback through nested invocation of words when an error
is detected at runtime.  Primitives: {\tt WALKBACK}.

\paragraph{The {\tt WORDSUSED} package.}
\label{`wordsused'}
Enables the collection of information on which words are used and not
used by a program, and the primitives that list words used and words
not used.  This facility allows you to determine, in the development
phase of an \atlast\ application, which packages are needed and which
can be safely dispensed with.  Primitives: {\tt WORDSUSED}, {\tt
WORDSUNUSED}.

\section{Benchmarks}

To give a rough idea of the kind of performance you can expect from
\atlast\ when it is pressed into service for compute-intensive tasks,
I tested it against C and AutoLISP with two benchmarks, both involving
the computation of square roots.

The first benchmark, {\tt CSQRT}, calculates the square root of 2
with the iterative Newton-Raphson algorithm used by AutoCAD's {\tt
HMATH.C} module, also used in the AutoLISP sample program
{\tt SQR.LSP}\@.  This benchmark is representative of extremely
compute-bound code which represents misuse of a macro language---any
such computation should normally be moved into a primitive written in
C\@.  Still, it's interesting to know what the worst case is.

The second benchmark, {\tt SSQRT}, is identical to {\tt CSQRT}, except
that the system math library's {\tt sqrt()} function is called instead
of one coded in the language under test.  Since all three languages
are calling the same underlying system function, this test
demonstrates relative performance in an environment still more
compute-bound than a typical macro language application, but one where the
language overhead is less than 100\%.  All of these benchmarks were 
run on a Sun 3/260 under SunOS 4.0.3, and listings of the benchmark
programs are given at the end of this paper.  The \atlast\ timings were
made on a version of \atlast\ compiled with the ``{\tt -O4 -f68881}''
flags, and stack and heap checking disabled in the \atlast\
configuration.  The C programs were also compiled with ``{\tt -O4
-f68881}'' flags, while the AutoLISP tests were run on a {\tt
NONPRODUCTION} version of {\tt Z.0.65} in which AutoLISP was built
with ``{\tt -O -f68881}''.  All timings in the following table have
been normalised so that the native C language times are 1.

\begin{center}
\begin{tabular}{lrrr}
 & \multicolumn{1}{c}{{\bf C}} &
   \multicolumn{1}{c}{{\bf ATLAST}} &
   \multicolumn{1}{c}{{\bf AutoLISP}} \\
{\tt CSQRT} & 1.00 & 7.41 & 67.08 \\
{\tt SSQRT} & 1.00 & 1.00 & 1.52  \\
\end{tabular}
\end{center}

\section{Summary and Conclusions}

Everything should be programmable.  {\em Everything!}  I have come to
the conclusion that to write almost any program in a closed manner is a
mistake that invites the expenditure of uncounted hours ``enhancing''
it over its life cycle.  Further tweaks, ``features,'' and ``fixes''
often result in a product so massive and incomprehensible that it becomes
unlearnable, unmaintainable, and eventually unusable.

Far better to invest the effort up front to create a product
flexible enough to be adapted at will, by its users, to their
immediate needs.  If the product is programmable in a portable, open
form, user extensions can be exchanged, compared, reviewed by the
product developer, and eventually incorporated into the mainstream of
the product.

It is far, far better to have thousands of creative users expanding
the scope of one's product in ways the original developers didn't
anticipate---in fact, working for the vendor without pay, than it is
to have thousands of frustrated users writing up wish list requests
that the vendor can comply with only by hiring people and paying them
to try to accommodate the perceived needs of the users.  Open
architecture and programmability not only benefits the user,
not only makes a product better in the technical and marketing sense,
but confers a direct economic advantage upon the vendor of such a
product---one mirrored in a commensurate disadvantage to the vendor
of a closed product.

The chief argument against programmability has been the extra
investment needed to create open products.  \atlast\ provides a way
of building open products in the same, or less, time than it takes to
construct closed ones.  Just as no C programmer in his right mind
would sit down and write his own buffered file I/O package when a
perfectly fine one was sitting in the library, why re-invent a macro
language or other parameterisation and programming facility when
there's one just sitting there that's as fast as native C code for all
but the most absurd misapplications, takes less than 51K with every
gew-gaw and optional feature at its command enabled all at once, is
portable to any machine that supports C by simply recompiling a single
file, and can be integrated into a typical application at a basic level
in less than 15 minutes?

Am I proposing that every application suddenly look like FORTH\@?
Of course not; no more than output from PostScript printers looks
like PostScript, or applications that run on 80386 processors resemble
80386 assembly language.  \atlast\ is an intermediate language, seen
only by those engaged in implementing and extending the product.  Even
then, \atlast\ is a chameleon which, with properly defined words, can
look like almost anything you like, even at the primitive level of the
interpreter.

Again and again, I have been faced with design situations where I
knew that I really needed programmability, but didn't have the time,
the memory, or the fortitude to face the problem squarely and solve it
the right way.  Instead, I ended up creating a kludge that continued
to burden me through time.  This is just a higher level manifestation
of the nightmares perpetrated by old-time programmers who didn't have
access to a proper dynamic memory allocator or linked list package.
Just because programmability is the magic smoke of computing doesn't
mean we should be spooked by the ghost in the machine or hesitant to
confer its power upon our customers.

Don't think of \atlast\ as FORTH\@.  Don't think of it as a language at
all.  The best way to think of \atlast\ is as a library routine that
gives you {\em programmability}, in the same sense other libraries
provide file access, window management, or graphics facilities.  The
whole concept of ``programmability in a can'' is odd---it took me two
years from the time I first thought about it in connection with The
Leto Protocol until I really got my end effector around it and crushed
it into submission.  I urge you to think about it, play with it, and
examine how it will be applied in the \atlast -enhanced programs I will
be demonstrating in the near future.

Open is better.  \atlast\ lets you build open programs in less time
than you used to spend writing closed ones.
Programs that inherit their open architecture from \atlast\ will share,
across the entire product line and among all hardware platforms that
support it, a common, clean, and efficient means of user
extensibility.  The potential benefits of this are immense.

{
\raggedleft \em
John Walker\\
Muir Beach, California\\
January 22--February 11, 1990\\
4072 lines of code\\
}

\onecolumn

\runhead{\huge \atlast\ Primitives: Alphabetical Reference}

\newcommand{\mf}{\bf}
\newcommand{\wline}[6]{\vspace{4pt}
                       \makebox[3cm][l]{\Large \tt #1}
                       \makebox[3cm][r]{#2} 
                       $\rightarrow$ 
                       \makebox[3cm][l]{#3}
                       \makebox[162pt][l]{\bf #4}
                       \makebox[2cm][r]{\small \tt #6}\\
\hfill\parbox[t]{225pt}{#5}\\
}
\begin{raggedright}
\wline{+}{n1 n2}{n3}{n3 = n1 $+$ n2}{Adds {\em n1} and {\em n2} and
    leaves sum on stack.}{}
\wline{-}{n1 n2}{n3}{n3 = n1 $-$ n2}{Subtracts {\em n2} from {\em n1} and
    leaves difference on stack.}{}
\wline{*}{n1 n2}{n3}{n3 = n1 $\times$ n2}{Multiplies {\em n1} and {\em n2} and
    leaves product on stack.}{}
\wline{/}{n1 n2}{n3}{n3 = n1 $\div$ n2}{Divides {\em n1} by {\em n2} and
    leaves quotient on stack.}{}
\wline{' {\em word}}{}{caddr}{Obtain compilation address}{Places the compilation
    address of the following word on the stack.}{}
\wline{,}{n}{}{Store in heap}{Reserves four bytes of heap space,
    initialising it to {\em n}.}{}
\wline{.}{n}{}{Print top of stack}{Prints the number on the top of the
    stack.}{CONIO}
\wline{.( {\em str}}{}{}{Print constant string}{Immediately prints the
string that follows in the input stream.}{CONIO}
\wline{.S}{}{}{Print stack}{Prints entire contents of stack.}{CONIO}
\wline{." {\em str}}{}{}{Print immediate string}{Prints the string
    literal that follows in line.}{CONIO}
\wline{: {\em w}}{}{}{Begin definition}{Begins compilation of a word
    named {\em w}.}{}
\wline{;}{}{}{End definition}{Ends compilation of word.}{}
\wline{<}{n1 n2}{flag}{Less than}{Returns $-1$ if {\em n1}$<${\em n2},
    0 otherwise.}{}
\wline{<=}{n1 n2}{flag}{Less than or equal}{Returns $-1$ if
    {\em n1}$\leq${\em n2}, 0 otherwise.}{}
\wline{<>}{n1 n2}{flag}{Not equal}{Returns $-1$ if
    {\em n1}$\neq${\em n2}, 0 otherwise.}{}
\wline{=}{n1 n2}{flag}{Equal}{Returns $-1$ if
    {\em n1}$=${\em n2}, 0 otherwise.}{}
\wline{>}{n1 n2}{flag}{Greater}{Returns $-1$ if
    {\em n1}$>${\em n2}, 0 otherwise.}{}
\wline{>=}{n1 n2}{flag}{Greater than or equal}{Returns $-1$ if
    {\em n1}$\geq${\em n2}, 0 otherwise.}{}
\wline{?}{addr}{}{Print indirect}{Prints the value at the address at
    the top of the stack.}{CONIO}
\wline{!}{n addr}{}{Store into address}{Stores the value {\em n} into
    the address {\em addr}.}{}
\wline{+!}{n addr}{}{Add indirect}{Adds {\em n} to the word at address
    {\em addr}.}{}
\wline{@}{addr}{n}{Load}{Loads the value at {\em addr} and leaves it
    at the top of the stack.}{}
\wline{[}{}{}{Set interpretive state}{Within a compilation, returns to
    the interpretive state.}{}
\wline{['] {\em word}}{}{caddr}{Push next word}{Places the compile
    address of the following word in a definition onto the stack.}{}
\wline{]}{}{}{End interpretive state}{Restore compile state after
    temporary interpretive state.}{}
\wline{0<}{n1}{flag}{Less than zero}{Returns $-1$ if {\em n1} less
    than zero, 0 otherwise.}{SHORTCUTC}
\wline{0<>}{n1}{flag}{Nonzero}{Returns $-1$ if {\em n1} is nonzero,
    0 otherwise.}{SHORTCUTC}
\wline{0=}{n1}{flag}{Equal to zero}{Returns $-1$ if {\em n1} is zero,
    0 otherwise.}{SHORTCUTC}
\wline{0>}{n1}{flag}{Greater than zero}{Returns $-1$ if {\em n1}
    greater than zero, 0 otherwise.}{SHORTCUTC}
\wline{1+}{n1}{n2}{Add one}{Adds one to top of stack.}{SHORTCUTA}
\wline{1-}{n1}{n2}{Subtract one}{Subtracts one from top of stack.}{SHORTCUTA}
\wline{2+}{n1}{n2}{Add two}{Adds two to top of stack.}{SHORTCUTA}
\wline{2-}{n1}{n2}{Subtract two}{Subtracts two from top of stack.}{SHORTCUTA}
\wline{2*}{n1}{n2}{Times two}{Multiplies the top of stack by
    two.}{SHORTCUTA}
\wline{2/}{n1}{n2}{Divide by two}{Divides top of stack by
    two.}{SHORTCUTA}
\wline{2!}{n1 n2 addr}{}{Store two words}{Stores the two words {\em
    n1} and {\em n2} at addresses {\em addr} and {\em
    addr}+4.}{DOUBLE}
\wline{2@}{addr}{n1 n2}{Load two words}{Places the two words starting
    at {\em addr} on the top of the stack}{DOUBLE}
\wline{2CONSTANT {\em x}}{n1 n2}{}{Double word constant}{Declares a
    double word constant {\em x}.  When {\em x} is executed, {\em n1}
    and {\em n2} are placed on the stack.}{DOUBLE}
\wline{2DROP}{n1 n2}{}{Double drop}{Discards the two top items from the
    stack.}{DOUBLE}
\wline{2DUP}{n1 n2}{n1 n2 n1 n2}{Duplicate two}{Duplicates the top two
    items on the stack.}{DOUBLE}
\wline{2OVER}{n1 n2 n3 n4}{n1 n2 n3 n4 n1 n2}{Double over}{Copies the
    second pair of items on the stack to the top of stack.}{DOUBLE}
\wline{2ROT}{n1 n2 n3 n4 n5 n6}{n3 n4 n5 n6 n1 n2}{Double
    rotate}{Rotates the third pair on the stack to the top, moving
    down the first and second pairs.}{DOUBLE}
\wline{2SWAP}{n1 n2 n3 n4}{n3 n4 n1 n2}{Double swap}{Swaps the first
    and second pairs on the stack.}{DOUBLE}
\wline{2VARIABLE {\em x}}{}{}{Double variable}{Creates a two cell (8
    byte) variable named {\em x}.  When {\em x} is executed, the
    address of the 8 byte area is placed on the stack.}{DOUBLE}
\wline{ABORT}{}{}{Abort}{Clears the stack and performs a {\tt
    QUIT}.}{}
\wline{ABORT" {\em str}}{}{}{Abort with message}{Prints the string
    literal that follows in line, then aborts, clearing all execution
    state to return to the interpreter.}{}
\wline{ABS}{n1}{n2}{${\mf n2}=|{\mf n1}|$}{Replaces top of stack with
    its absolute value.}{}
\wline{ACOS}{f1}{f2}{${\mf f2}=\arccos {\mf f1}$}{Replaces floating
    point top of stack with its arc cosine.}{MATH}
\wline{AGAIN}{}{}{Indefinite loop}{Marks the end of an indefinite loop
   opened by the matching {\tt BEGIN}.}{}
\wline{ALLOT}{n}{}{Allocate heap}{Allocates {\em n} bytes of heap
    space.  The space allocated is rounded to the next higher multiple
    of 4.}{}
\wline{AND}{n1 n2}{n3}{Bitwise AND}{Stores the bitwise AND of {\em n1}
    and {\em n2} on the stack.}{}
\wline{ARRAY {\em x}}{s$_1$ s$_2$ \ldots\ s$_n$ n esize}{}{Declare
    array}{Declares an array {\em x} of elements of {\em esize}
    bytes each with {\em n} subscripts, each ranging from 0 to {\em
    s}$_n-1$}{ARRAY}
\wline{ASIN}{f1}{f2}{${\mf f2}=\arcsin {\mf f1}$}{Replaces floating
    point top of stack with its arc sine.}{MATH}
\wline{ATAN}{f1}{f2}{${\mf f2}=\arctan {\mf f1}$}{Replaces floating
    point top of stack with its arc tangent.}{MATH}
\wline{ATAN2}{f1 f2}{f3}{${\mf f3}=\arctan {\mf f1}/{\mf f2}$}{Replaces
    the two floating point numbers on the top of the stack with
    the arc tangent of their quotient, properly handling zero
    denominators.}{MATH}
\wline{BEGIN}{}{}{Begin loop}{Begins an indefinite loop.  The end of
    the loop is marked by the matching {\tt AGAIN}, {\tt REPEAT}, or
    {\tt UNTIL}.}{}
\wline{BODY>}{pfa}{cfa}{Body to word}{Given body address of word,
    return the compile address of the word.}{DEFFIELDS}
\wline{>BODY}{cfa}{pfa}{Body address}{Given the compile address of a
    word, return its body (parameter) address.}{}
\wline{BRANCH}{}{}{Branch}{Jump to the address that follows in
    line.}{}
\wline{?BRANCH}{flag}{}{Conditional branch}{If the top of stack is
    zero, jump to the address which follows in line.  Otherwise 
    skip the address and continue execution.}{}
\wline{C!}{n addr}{}{Store byte}{The 8 bit value {\em n} is stored in
    the byte at address {\em addr}.}{}
\wline{C@}{addr}{n}{Load byte}{The byte at address {\em addr} is
    placed on the top of the stack.}{}
\wline{C,}{n}{}{Compile byte}{The 8 bit value {\em n} is stored in the
    next free byte of the heap and the heap pointer is incremented by
    one.}{}
\wline{C=}{}{}{Align heap}{The heap allocation pointer is adjusted to
    the next four byte boundary.  This must be done following a
    sequence of {\tt C,} operations.}{}
\wline{CLEAR}{}{}{Clear stack}{All items on the stack are
    discarded.}{}
\wline{COMPARE}{s1 s2}{n}{Compare strings}{The two strings whose
    addresses are given by {\em s1} and {\em s2} are compared.
    If {\em s1} is less than {\em s2}, $-1$ is returned; if
    {\em s1} is greater than {\em s2}, 1 is returned.  If
    {\em s1} and {\em s2} are equal, 0 is returned.}{STRING}
\wline{COMPILE {\em w}}{}{}{Compile word}{Adds the compile address of
   the word that follows in line to the definition currently being
   compiled.}{COMPILERW}
\wline{[COMPILE] {\em word}}{}{}{Compile immediate word}{Compiles the
    address of {\em word}, even if {\em word} is marked {\tt
    IMMEDIATE}.}{COMPILERW}
\wline{CONSTANT {\em x}}{n}{}{Declare constant}{Declares a constant
    named {\em x}.  When {\em x} is executed, the value {\em n} will
    be left on the stack.}{}
\wline{COS}{f1}{f2}{Cosine}{The floating point value on the top of the stack is
    replaced by its cosine.}{MATH}
\wline{CR}{}{}{Carriage return}{The standard output stream is advanced
    to the first character of the next line.}{CONIO}
\wline{CREATE}{}{}{Create object}{Create an object, given the name
    which appears next in the input stream, with a default action of
    pushing the parameter field address of the object when executed.
    No storage is allocated; normally the parameter field will be
    allocated and initialised by the defining word code that follows
    the {\tt CREATE}.}{}
\wline{DEPTH}{}{n}{Stack depth}{Returns the number of items on the
    stack before {\tt DEPTH} was executed.}{}
\wline{DO}{limit n}{}{Definite loop}{Executes the loop from the
    following word to the matching {\tt LOOP} or {\tt +LOOP} until
    {\em n} increments past the boundary between {\em limit}$-1$ and
    {\em limit}.  Note that the loop is always executed at least once
    (see {\tt ?DO} for an alternative to this).}{}
\wline{?DO}{limit n}{}{Conditional loop}{If {\em n} equals {\em
    limit}, skip immediately to the matching {\tt LOOP} or {\tt
    +LOOP}.  Otherwise, enter the loop, which is thenceforth treated
    as a normal {\tt DO} loop.}{}
\wline{DOES>}{}{}{Run-time action}{Sets the run-time action of a word
    created by the last {\tt CREATE} to the code that follows.  When
    the word is executed, its body address is pushed on the
    stack, then the code that follows the {\tt DOES>} will be
    executed.}{}
\wline{DROP}{n}{}{Discard top of stack}{Discards the value at the top
    of the stack.}{}
\wline{DUP}{n}{n n}{Duplicate}{Duplicates the value at the top of the
    stack.}{}
\wline{?DUP}{n}{0 / n n}{Conditional duplicate}{If top of stack is
    nonzero, duplicate it.  Otherwise leave zero on top of stack.}{}
\wline{ELSE}{}{}{Else}{Used in an {\tt IF}---{\tt ELSE}---{\tt THEN}
    sequence, delimits the code to be executed if the if-condition
    was false.}{}
\wline{EXECUTE}{addr}{}{Execute word}{Executes the word with compile
    address {\em addr}.}{}
\wline{EXIT}{}{}{Exit definition}{Exit from the current definition
   immediately.  Note that {\tt EXIT} cannot be used within
   a {\tt DO}---{\tt LOOP}; use {\tt LEAVE} instead.}{}
\wline{EXP}{f1}{f2}{${\mf f2}={\mf e}^{\mf f1}$}{The floating point value on
   the top of the stack is replaced by its natural
   antilogarithm.}{MATH}
\wline{F+}{f1 f2}{f3}{${\mf f3}={\mf f1}+{\mf f2}$}{The two
    floating point values on the top of the stack are added and their
    sum is placed on the top of the stack.}{REAL}
\wline{F-}{f1 f2}{f3}{${\mf f3}={\mf f1}-{\mf f2}$}{The
    floating point value {\em f2} is subtracted from the floating
    point value {\em f1} and the result
    is placed on the top of the stack.}{REAL}
\wline{F*}{f1 f2}{f3}{${\mf f3}={\mf f1}\times{\mf f2}$}{The two
    floating point values on the top of the stack are multiplied and their
    product is placed on the top of the stack.}{REAL}
\wline{F/}{f1 f2}{f3}{${\mf f3}={\mf f1}\div{\mf f2}$}{The
    floating point value {\em f1} is divided by the floating point
    value {\em f2} and the
    quotient is placed on the top of the stack.}{REAL}
\wline{F.}{f}{}{Print floating point}{The floating point value on the
    top of the stack is printed.}{REAL}
\wline{F<}{f1 f2}{flag}{Floating less than}{The top of stack is set to
    $-1$ if {\em f1} is less than
    {\em f2} and 0 otherwise.}{REAL}
\wline{F<=}{f1 f2}{flag}{Floating less than or equal}{The top of stack is
    set to
    $-1$ if {\em f1} is less than or equal to
    {\em f2} and 0 otherwise.}{REAL}
\wline{F<>}{f1 f2}{flag}{Floating not equal}{The top of stack is set to
    $-1$ if {\em f1} is not equal to
    {\em f2} and 0 otherwise.}{REAL}
\wline{F=}{f1 f2}{flag}{Floating equal}{The top of stack is set to
    $-1$ if {\em f1} is equal to
    {\em f2} and 0 otherwise.}{REAL}
\wline{F>}{f1 f2}{flag}{Floating greater than}{The top of stack is set to
    $-1$ if {\em f1} is greater than
    {\em f2} and 0 otherwise.}{REAL}
\wline{F>=}{f1 f2}{flag}{Floating greater than or equal}{The top of stack
    is set to
    $-1$ if {\em f1} is greater than or equal to
    {\em f2} and 0 otherwise.}{REAL}
\wline{FABS}{f1}{f2}{${\mf f2}=|{\mf f1}|$}{Replaces floating point
    top of stack with its absolute value.}{}
\wline{FCLOSE}{file}{}{Close file}{The specified file is
    closed.}{FILEIO}
\wline{FDELETE}{s1}{flag}{Delete file}{The file named by the string
    {\em s1} is deleted.  If the file was successfully deleted, $-1$
    is returned.   Otherwise, 0 is returned.}{FILEIO}
\wline{FGETC}{file}{char}{Read next character}{The next byte is read
    from the specified {\em file} and placed on the top of the stack.
    If end of file is encountered, $-1$ is returned.}{FILEIO}
\wline{FGETS}{file string}{flag}{Read string}{The next text line
    (limited to a maximum of 132 characters) is read from {\em file}
    and stored into the buffer at {\em string}.  Input lines are
    recognised in all the end of line conventions accepted by
    AutoCAD\@.  The end of line delimiter is deleted from the input
    line and is not stored in the {\em string}.
    If end of file is encountered 0 is returned; otherwise
    $-1$ is placed on the top of the stack.}{FILEIO}
\wline{FILE {\em f}}{}{}{Declare file}{A file descriptor named {\em f}
    is declared.  This descriptor may subsequently be associated
    with a file with {\tt FOPEN}.}{FILEIO}
\wline{FIND}{s}{word flag}{Look up word}{The word with name given
    by the string {\em s} is looked up in the dictionary.  If a
    definition if not found, {\em word} will be left as the address
    of the string and {\em flag} will be set to zero.  If the word is
    present in the dictionary, its compilation address is placed on
    the stack, followed by a {\em flag} that is 1 if the word is
    marked for immediate execution and $-1$ otherwise.}{DEFFIELDS}
\wline{FIX}{f}{n}{Floating to integer}{The floating point number
    on the top of the stack is replaced by the integer obtained by
    truncating its fractional part.}{REAL}
\wline{(FLIT)}{}{f}{Push floating point literal}{Pushes the
    floating point literal that follows in line onto the top of
    the stack.}{REAL}
\wline{FLOAD}{file}{stat}{Load file}{The source program starting at
    the current position in {\em file} is loaded as if its text
    appeared at the current character position in the input stream.
    The status resulting from the evaluation is left on the stack,
    zero if normal, negative in case of error.}{FILEIO}
\wline{FLOAT}{n}{f}{Integer to floating}{The integer value on the top
    of the stack is replaced by the equivalent floating point
    value.}{REAL}
\wline{FMAX}{f1 f2}{f3}{Floating point maximum}{The greater of the two
    floating point values on the top of the stack is placed on the
    top of the stack.}{FLOAT}
\wline{FMIN}{f1 f2}{f3}{Floating point minimum}{The lesser of the two
    floating point values on the top of the stack is placed on the
    top of the stack.}{FLOAT}
\wline{FNEGATE}{f1}{f2}{${\mf f2} = -{\mf f1}$}{The negative of the
    floating point value on the top of the stack replaces the floating
    point value there.}{FLOAT}
\wline{FOPEN}{fname fmodes file}{flag}{File open}{The previously
    declared {\em file} is opened with the specified file name {\em
    fname} given by the string address on the stack in the mode
    given by {\em fmodes}.  The bits in {\em fmodes} are 1 for read,
    2 for write, 4 for binary, and 8 to create a new
    file.  If the file is opened successfully, $-1$ is returned;
    otherwise 0 is returned.  The Unix standard streams, {\tt STDIN},
    {\tt STDOUT}, and {\tt STDERR} are predefined and automatically
    opened.}{FILEIO}
\wline{FORGET {\em w}}{}{}{Forget word}{The most recent definition of
    word {\em w} is deleted, along with all words declared more
    recently than the named word.}{}
\wline{FPUTC}{char file}{stat}{Write character}{The character {\em
    char} is written to {\em file}.  If the character is written
    successfully, {\em char} is returned; otherwise $-1$ is
    returned.}{FILEIO}
\wline{FPUTS}{s file}{flag}{Write string}{The string {\em s} is
    written to {\em file}, followed by the end of line delimiter used
    on this system.  If the line is written successfully, $-1$ is
    returned; otherwise 0 is returned.}{FILEIO}
\wline{FREAD}{file len buf}{length}{Read file}{{\em Len} bytes are
   read into buffer {\em buf} from {\em file}.  The number of
   bytes actually read is returned on the top of the stack.}{FILEIO}
\wline{FSEEK}{offset base file}{}{Set file position}{The current
    position of {\em file} is set to {\em offset}, relative to the
    specified {\em base}: if 0, the beginning of the file; if 1, the
    current file position; if 2, the end of file.}{FILEIO}
\wline{FSTRFORM}{f format str}{}{Floating point edit}{Edits a floating
    point number {\em f} into string {\em str}, using the {\tt
    sprintf} format given by the string {\em format}.}{REAL}
\wline{FTELL}{file}{pos}{File position}{Returns the current byte
    position {\em pos} for file {\em file}.}{FILEIO}
\wline{FWRITE}{len buf file}{length}{File write}{Writes {\em len}
    bytes from the buffer at address {\em buf} to {\em file}.
    The number of bytes written is returned on the top of the
    stack.}{FILEIO}
\wline{HERE}{}{addr}{Heap address}{The current heap allocation address
    is placed on the top of the stack.}{}
\wline{I}{}{n}{Inner loop index}{The index of the innermost {\tt
    DO}---{\tt LOOP} is placed on the stack.}{}
\wline{IF}{flag}{}{Conditional statement}{If {\em flag} is nonzero,
    the following statements are executed.  Otherwise, execution
    resumes after the matching {\tt ELSE} clause, if any, or after the
    matching {\tt THEN}.}{}
\wline{IMMEDIATE}{}{}{Mark immediate}{The most recently defined word
    is marked for immediate execution; it will be executed even if
    entered in compile state.}{}
\wline{J}{}{n}{Outer loop index}{The loop index of the next to
    innermost {\tt DO}---{\tt LOOP} is placed on the stack.}{}
\wline{L>NAME}{lfa}{nfa}{Link to name field}{Given the link field
    address of a word on the top of the stack, its name pointer field
    address is returned.}{DEFFIELDS}
\wline{LEAVE}{}{}{Exit {\tt DO}---{\tt LOOP}}{The innermost {\tt
    DO}---{\tt LOOP} is immediately exited.  Execution resumes after
    the {\tt LOOP} statement marking the end of the loop.}{}
\wline{LINK>}{lfa}{cfa}{Link field to compile address}{Given the link field
    address of a word on the top of the stack, the compile
    address of the word is returned.}{DEFFIELDS}
\wline{>LINK}{cfa}{lfa}{Link address}{Given the compile address of a
    word, return its link field address.}{DEFFIELDS}
\wline{(LIT)}{}{n}{Push literal}{Pushes the integer literal that
    follows in line onto the top of the stack.}{}
\wline{LITERAL}{n}{}{Compile literal}{Compiles the value on the top of
    the stack into the current definition.  When the definition is
    executed, that value will be pushed onto the top of the
    stack.}{COMPILERW}
\wline{LOG}{f1}{f2}{${\mf f2}=\ln {\mf f1}$}{The floating point value on
   the top of the stack is replaced by its natural logarithm.}{MATH}
\wline{LOOP}{}{}{Increment loop index}{Adds one to the index of 
    the active loop.  If the limit is reached, the loop is exited.
    Otherwise, another iteration is begun.}{}
\wline{+LOOP}{n}{}{Add to loop index}{Adds {\em n} to the index of 
    the active loop.  If the limit is reached, the loop is exited.
    Otherwise, another iteration is begun.}{}
\wline{<MARK}{}{addr}{Backward jump mark}{Saves the current
    compilation address on the stack.}{COMPILERW}
\wline{>MARK}{}{addr}{Forward mark}{Compiles a place-holder offset for
    a forward jump and saves its address for later backpatching on
    the stack.}{COMPILERW}
\wline{MAX}{n1 n2}{n3}{Maximum}{The greater of {\em n1} and {\em n2} is
    left on the top of the stack.}{}
\wline{MEMSTAT}{}{}{Print memory status}{The current and maximum
    memory usage so far are printed on standard output.  The sizes
    allocated for the stack, return stack, and heap are edited, as
    well as the percentage in use.}{MEMSTAT}
\wline{MIN}{n1 n2}{n3}{Minimum}{The lesser of {\em n1} and {\em n2} is
    left on the top of the stack.}{}
\wline{MOD}{n1 n2}{n3}{Modulus (remainder)}{The remainder when {\em
    n1} is divided by {\em n2} is left on the top of the stack.}{}
\wline{/MOD}{n1 n2}{n3 n4}{n3 = n1 $\bmod$ n2, n4 = n1 $\div$ n2}{
    Divides {\em n1} by {\em n2} and leaves quotient on top of
    stack, remainder as next on stack.}{}
\wline{N>LINK}{nfa}{lfa}{Name to link field}{Given the name field
    pointer address of a word on the top of the stack, leaves the link
    field address of the word on the top of stack.}{DEFFIELDS}
\wline{>NAME}{cfa}{nfa}{Name address}{Given the compile address of a
    word, return its name pointer field address.}{DEFFIELDS}
\wline{NAME>}{nfa}{cfa}{Name field to compile address}{Given the
    address of the name pointer field of a word on the top of the
    stack, leaves the compile address of the word on the top of the
    stack.}{DEFFIELDS}
\wline{NAME>S!}{nfa string}{}{Get name field}{Stores the
    name field of the word pointed to by {\em nfa} into {\em
    string}.}{DEFFIELDS} 
\wline{NEGATE}{n1}{n2}{${\mf n2}=-{\mf n1}$}{Negates the value on the
    top of the stack.}{}
\wline{(NEST)}{}{}{Invoke word}{Pushes the instruction pointer onto
    the return stack and sets the instruction pointer to the next word
    in line.}{}
\wline{NOT}{n1}{n2}{Logical not}{Inverts the bits in the value on the
    top of the stack.  This performs logical negation for truth
    values of $-1$ (True) and $0$ (False).}{}
\wline{OR}{n1 n2}{n3}{Bitwise OR}{Stores the bitwise OR of {\em n1}
    and {\em n2} on the stack.}{}
\wline{OVER}{n1 n2}{n1 n2 n1}{Duplicate second item}{The second item
    on the stack is copied to the top.}{}
\wline{PICK}{\ldots n$_2$ n$_1$ n$_0$ index}{\ldots n$_0$ n$_{index}$}{Pick
    item
    from stack}{The {\em index}th stack item is copied to the top of
    the stack.  The top of stack has {\em index} 0, the second item
    {\em index} 1, and so on.}{}
\wline{POW}{f1 f2}{f3}{${\mf f3}={\mf f1}^{\mf f2}$}{The second
    floating point value on the stack is taken to the power of the
    top floating point stack value and the result is left on the
    top of the stack.}{MATH}
\wline{QUIT}{}{}{Quit execution}{The return stack is cleared and
    control is returned to the interpreter.  The stack is not
    disturbed.}{}
\wline{>R}{n}{}{To return stack}{Removes the top item from the stack
    and pushes it onto the return stack.}{}
\wline{R>}{}{n}{From return stack}{The top value is removed from the
    return stack and pushed onto the stack.}{}
\wline{R@}{}{n}{Fetch return stack}{The top value on the return stack
    is pushed onto the stack.  The value is not removed from the
    return stack.}{}
\wline{REPEAT}{}{}{Close {\tt BEGIN}---{\tt WHILE}---{\tt REPEAT}
    loop}{Another iteration of the current {\tt BEGIN}---{\tt
    WHILE}---{\tt REPEAT} loop having been completed, execution
    continues after the matching {\tt BEGIN}.}{}
\wline{<RESOLVE}{addr}{}{Backward jump resolve}{Compiles the address
    saved by the matching {\tt <MARK}.}{COMPILERW}
\wline{>RESOLVE}{addr}{}{Forward jump resolve}{Backpatches the address left
    by the matching {\tt >MARK} to jump to the next word to be
    compiled.}{COMPILERW}
\wline{ROLL}{\ldots n$_2$ n$_1$ n$_0$ index}{\ldots n$_0$
    n$_{index}$}{Rotate {\em index}th item to top}{The stack item selected
    by {\em index}, with 0 designating the top of stack, 1 the
    second item, and so on, is moved to the top of the stack.  The
    intervening stack items are moved down one item.}{}
\wline{ROT}{n1 n2 n3}{n2 n3 n1}{Rotate 3 items}{The third item on the
    stack is placed on the top of the stack and the second and first
    items are moved down.}{}
\wline{-ROT}{n1 n2 n3}{n3 n1 n2}{Reverse rotate}{Moves the top of
    stack to the third item, moving the third and second items up.}{}
\wline{S!}{s1 s2}{}{Store string}{The string at address {\em s1} is
    copied into the string at {\em s2}.}{STRING}
\wline{S+}{s1 s2}{}{String concatenate}{The string at address {\em s1}
    is concatenated to the string at address {\em s2}.}{STRING}
\wline{S>NAME!}{string nfa}{}{Store name field}{Stores the {\em string}
    into the name field of the word given by name pointer
    field {\em nfa}.}{DEFFIELDS}
\wline{SHIFT}{n1 n2}{n3}{Shift n1 by n2 bits}{The value {\em n1} is
    logically shifted the number of bits specified by {\em n2},
    left if {\em n2} is positive and right if {\em n2} is
    negative.  Zero bits are shifted into vacated bits.}{}
\wline{SIN}{f1}{f2}{Sine}{The floating point value on the top of the stack is
    replaced by its sine.}{MATH}
\wline{SQRT}{f1}{f2}{${\mf f2}=\sqrt{{\mf f1}}$}{The floating point
    value on the top of the stack is replaced by its square
    root.}{MATH}
\wline{STATE}{}{addr}{System state variable}{The address of the system
    state variable is pushed on the stack.  The state is zero if
    interpreting, nonzero if compiling.}{}
\wline{STRCAT}{s1 s2}{}{String concatenate}{The string at address {\em s1}
    is concatenated to the string at address {\em s2}.}{STRING}
\wline{STRCHAR}{s1 s2}{}{String character search}{The string at
    address {\em s1} is searched for the first occurrence of the first
    character of string {\em s2}.  If that character appears nowhere
    in {\em s1}, 0 is returned.  Otherwise, the address of the first
    occurrence in {\em s1} is left on the top of the stack.}{STRING}
\wline{STRCMP}{s1 s2}{n}{String compare}{The string at address {\em s1}
    is compared to the string at address {\em s2}.  If {\em s1} is
    less than {\em s2}, $-1$ is returned.  If {\em s1} and {\em s2}
    are equal, 0 is returned.  If {\em s1} is greater than {\em s2}, 1
    is returned.}{STRING}
\wline{STRCPY}{s1 s2}{}{Store string}{The string at address {\em s1} is
    copied into the string at {\em s2}.}{STRING}
\wline{STRFORM}{n format str}{}{Integer edit}{Edits the
    number {\em n} into string {\em str}, using the {\tt
    sprintf} format given by the string {\em format}.  Note: the
    reference to the number in the format must be as a {\tt long}
    value, for example {\tt "\%ld"}.}{STRING}
\wline{STRING {\em x}}{size}{}{Declare string}{Declares a string named
    {\em x} of a maximum of {\em size}$-1$ characters.}{STRING}
\wline{STRINT}{s1}{s2 n}{String to integer}{Scans an integer
    from {\em s1}.  The integer scanned is placed on the top of
    the stack and the address of the character that terminated the
    scan is stored as the next item on the stack.}{STRING}
\wline{STRLEN}{s}{n}{String length}{The length of string {\em s} is
    placed on the top of the stack.}{STRING}
\wline{(STRLIT)}{}{s}{String literal}{Pushes the address of the string literal
    that follows in line onto the stack.}{STRING}
\wline{STRREAL}{s1}{s2 f}{String to real}{Scans a floating point
    number from {\em s1}.  The floating point number scanned is placed
    on the top of the stack and the address of the character that
    terminated the scan is stored as the next item on the
    stack.}{STRING}
\wline{SUBSTR}{s1 start length s2}{}{Extract substring}{The substring
    of string {\em s1} that begins at character {\em start}, with the
    first character numbered 0, extending for {\em length} characters,
    with $-1$ designating all characters to the end of string, is stored
    into the string {\em s2}.}{STRING}
\wline{SWAP}{n1 n2}{n2 n1}{Swap top two items}{The top two stack items
    are interchanged.}{}
\wline{SYSTEM}{s}{n}{Execute system command}{The operating system
    command given in the string {\em s} is passed to the system's
    command interpreter (shell).  The system result status returned
    after the command completes is left on the top of the
    stack.}{SYSTEM}
\wline{TAN}{f1}{f2}{Tangent}{The floating point value on the top of
    the stack is replaced by its tangent.}{MATH}
\wline{THEN}{}{}{End if}{Used in an {\tt IF}---{\tt ELSE}---{\tt THEN}
    sequence, marks the end of the conditional statement.}{}
\wline{TRACE}{n}{}{Trace mode}{If {\em n} is nonzero, trace mode is
    enabled.  If {\em n} is zero, trace mode is turned off.}{TRACE}
\wline{TYPE}{s}{}{Print string}{The string at address {\em s} is
    printed on standard output.}{CONIO}
\wline{UNTIL}{flag}{}{End {\tt BEGIN}---{\tt UNTIL} loop}{If {\em
    flag} is zero, the loop continues execution at the word following
    the matching {\tt BEGIN}\@.  If {\em flag} is nonzero, the loop is
    exited and the word following the {\tt UNTIL} is executed.}{}
\wline{VARIABLE {\em x}}{}{}{Declare variable}{A variable named {\em
    x} is declared and its value is set to zero.  When {\em x}
    is executed, its address will be placed on the stack.  Four bytes
    are reserved on the heap for the variable's value.}{}
\wline{WALKBACK}{n}{}{Walkback mode}{If {\em n} is nonzero, a walkback
    trace through active words will be performed whenever an error
    occurs during execution.  If {\em n} is zero, the walkback is
    suppressed.}{WALKBACK}
\wline{WHILE}{flag}{}{Decide {\tt BEGIN}---{\tt WHILE}---{\tt REPEAT}
    loop}{If {\em flag} is nonzero, execution continues after the {\tt
    WHILE}\@.  If {\em flag} is zero, the loop is exited and execution
    resumed after the {\tt REPEAT} that marks the end of the loop.}{}
\wline{WORDS}{}{}{List words defined}{Defined words are listed, from
    the most recently defined to the first defined.  If the system
    supports keystroke trapping, pressing any key will pause the
    display of defined words; pressing carriage return will abort the
    listing---any other key resumes it.  On other systems, only the 20
    most recently defined words are listed.}{CONIO}
\wline{WORDSUSED}{}{}{List words used}{The words used by this program
    are listed on standard output.  If the system supports keystroke
    trapping, the listing may be aborted by pressing a key while the
    output is in progress.  The words used report is useful in
    configuring a custom version of \atlast\ that includes just the words
    needed by the program it executes.}{WORDSUSED}
\wline{WORDSUNUSED}{}{}{List words not used}{The words not used by
    this program are listed on standard output.  If the system
    supports keystroke trapping, the listing may be aborted by
    pressing a key while the output is in progress.  The words not used
    report is useful in configuring a custom version of \atlast\ that
    includes just the words needed by the program it
    executes.}{WORDSUSED}
\wline{XOR}{n1 n2}{n3}{Bitwise exclusive OR}{Stores the bitwise
    exclusive or of {\em n1} and {\em n2} on the stack.}{}
\wline{(XDO)}{limit n}{}{Execute loop}{At runtime, enters a loop that
    will step until {\em n} increments and becomes equal to {\em limit}}{}
\wline{(X?DO)}{limit n}{}{Execute conditional loop}{At runtime, tests
    if {\em n} equals {\em limit}.  If so, skips until the matching {\tt
    LOOP} or {\tt +LOOP}.  Otherwise, enters the loop.}{}
\wline{(XLOOP)}{}{}{Increment loop index}{At runtime, adds one to the
    index of the active loop and exits if equal to the limit.
    Otherwise returns to the matching {\tt DO} or {\tt ?DO}.}{}
\wline{(+XLOOP)}{incr}{}{Add to loop index}{At runtime, increments
    the loop index by the top of stack.  If the loop is not done, begins
    the next iteration.}{}

\end{raggedright}

\clearpage

\runhead{\huge \atlast\ Primitives: Alphabetical Summary}

\renewcommand{\mf}{\rm}
\renewcommand{\wline}[6]{{\Large \tt #1} \>
 \makebox[110pt][r]{#2} \> $\rightarrow$ \>  \makebox[3cm][l]{#3} \> {#4} \\}
\begin{tabbing}
dlkjldjdlkjdlkjdl;jdj \= \makebox[110pt][r]{x} \= $\rightarrow$ \=
\makebox[3cm][l]{x} \= dlkjdlkdjdljkdljk \=
\kill
\wline{+}{n1 n2}{n3}{n3 = n1 $+$ n2}{Adds {\em n1} and {\em n2} and
    leaves sum on stack.}{}
\wline{-}{n1 n2}{n3}{n3 = n1 $-$ n2}{Subtracts {\em n2} from {\em n1} and
    leaves difference on stack.}{}
\wline{*}{n1 n2}{n3}{n3 = n1 $\times$ n2}{Multiplies {\em n1} and {\em n2} and
    leaves product on stack.}{}
\wline{/}{n1 n2}{n3}{n3 = n1 $\div$ n2}{Divides {\em n1} by {\em n2} and
    leaves quotient on stack.}{}
\wline{' {\em word}}{}{caddr}{Obtain compilation address}{Places the compilation
    address of the following word on the stack.}{}
\wline{,}{n}{}{Store in heap}{Reserves four bytes of heap space,
    initialising it to {\em n}.}{}
\wline{.}{n}{}{Print top of stack}{Prints the number on the top of the
    stack.}{CONIO}
\wline{.( {\em str}}{}{}{Print constant string}{Immediately prints the
string that follows in the input stream.}{CONIO}
\wline{.S}{}{}{Print stack}{Prints entire contents of stack.}{CONIO}
\wline{." {\em str}}{}{}{Print immediate string}{Prints the string
    literal that follows in line.}{CONIO}
\wline{: {\em w}}{}{}{Begin definition}{Begins compilation of a word
    named {\em w}.}{}
\wline{;}{}{}{End definition}{Ends compilation of word.}{}
\wline{<}{n1 n2}{flag}{Less than}{Returns $-1$ if {\em n1}$<${\em n2},
    0 otherwise.}{}
\wline{<=}{n1 n2}{flag}{Less than or equal}{Returns $-1$ if
    {\em n1}$\leq${\em n2}, 0 otherwise.}{}
\wline{<>}{n1 n2}{flag}{Not equal}{Returns $-1$ if
    {\em n1}$\neq${\em n2}, 0 otherwise.}{}
\wline{=}{n1 n2}{flag}{Equal}{Returns $-1$ if
    {\em n1}$=${\em n2}, 0 otherwise.}{}
\wline{>}{n1 n2}{flag}{Greater}{Returns $-1$ if
    {\em n1}$>${\em n2}, 0 otherwise.}{}
\wline{>=}{n1 n2}{flag}{Greater than or equal}{Returns $-1$ if
    {\em n1}$\geq${\em n2}, 0 otherwise.}{}
\wline{?}{addr}{}{Print indirect}{Prints the value at the address at
    the top of the stack.}{CONIO}
\wline{!}{n addr}{}{Store into address}{Stores the value {\em n} into
    the address {\em addr}.}{}
\wline{+!}{n addr}{}{Add indirect}{Adds {\em n} to the word at address
    {\em addr}.}{}
\wline{@}{addr}{n}{Load}{Loads the value at {\em addr} and leaves it
    at the top of the stack.}{}
\wline{[}{}{}{Set interpretive state}{Within a compilation, returns to
    the interpretive state.}{}
\wline{['] {\em word}}{}{caddr}{Push next word}{Places the compile
    address of the following word in a definition onto the stack.}{}
\wline{]}{}{}{End interpretive state}{Restore compile state after
    temporary interpretive state.}{}
\wline{0<}{n1}{flag}{Less than zero}{Returns $-1$ if {\em n1} less
    than zero, 0 otherwise.}{SHORTCUTC}
\wline{0<>}{n1}{flag}{Nonzero}{Returns $-1$ if {\em n1} is nonzero,
    0 otherwise.}{SHORTCUTC}
\wline{0=}{n1}{flag}{Equal to zero}{Returns $-1$ if {\em n1} is zero,
    0 otherwise.}{SHORTCUTC}
\wline{0>}{n1}{flag}{Greater than zero}{Returns $-1$ if {\em n1}
    greater than zero, 0 otherwise.}{SHORTCUTC}
\wline{1+}{n1}{n2}{Add one}{Adds one to top of stack.}{SHORTCUTA}
\wline{1-}{n1}{n2}{Subtract one}{Subtracts one from top of stack.}{SHORTCUTA}
\wline{2+}{n1}{n2}{Add two}{Adds two to top of stack.}{SHORTCUTA}
\wline{2-}{n1}{n2}{Subtract two}{Subtracts two from top of stack.}{SHORTCUTA}
\wline{2*}{n1}{n2}{Times two}{Multiplies the top of stack by
    two.}{SHORTCUTA}
\wline{2/}{n1}{n2}{Divide by two}{Divides top of stack by
    two.}{SHORTCUTA}
\wline{2!}{n1 n2 addr}{}{Store two words}{Stores the two words {\em
    n1} and {\em n2} at addresses {\em addr} and {\em
    addr}+4.}{DOUBLE}
\wline{2@}{addr}{n1 n2}{Load two words}{Places the two words starting
    at {\em addr} on the top of the stack}{DOUBLE}
\wline{2CONSTANT {\em x}}{n1 n2}{}{Double word constant}{Declares a
    double word constant {\em x}.  When {\em x} is executed, {\em n1}
    and {\em n2} are placed on the stack.}{DOUBLE}
\wline{2DROP}{n1 n2}{}{Double drop}{Discards the two top items from the
    stack.}{DOUBLE}
\wline{2DUP}{n1 n2}{n1 n2 n1 n2}{Duplicate two}{Duplicates the top two
    items on the stack.}{DOUBLE}
\wline{2OVER}{n1 n2 n3 n4}{n1 n2 n3 n4 n1 n2}{Double over}{Copies the
    second pair of items on the stack to the top of stack.}{DOUBLE}
\wline{2ROT}{n1 n2 n3 n4 n5 n6}{n3 n4 n5 n6 n1 n2}{Double
    rotate}{Rotates the third pair on the stack to the top, moving
    down the first and second pairs.}{DOUBLE}
\wline{2SWAP}{n1 n2 n3 n4}{n3 n4 n1 n2}{Double swap}{Swaps the first
    and second pairs on the stack.}{DOUBLE}
\wline{2VARIABLE {\em x}}{}{}{Double variable}{Creates a two cell (8
    byte) variable named {\em x}.  When {\em x} is executed, the
    address of the 8 byte area is placed on the stack.}{DOUBLE}
\wline{ABORT}{}{}{Abort}{Clears the stack and performs a {\tt
    QUIT}.}{}
\wline{ABORT" {\em str}}{}{}{Abort with message}{Prints the string
    literal that follows in line, then aborts, clearing all execution
    state to return to the interpreter.}{}
\wline{ABS}{n1}{n2}{${\mf n2}=|{\mf n1}|$}{Replaces top of stack with
    its absolute value.}{}
\wline{ACOS}{f1}{f2}{${\mf f2}=\arccos {\mf f1}$}{Replaces floating
    point top of stack with its arc cosine.}{MATH}
\wline{AGAIN}{}{}{Indefinite loop}{Marks the end of an indefinite loop
   opened by the matching {\tt BEGIN}.}{}
\wline{ALLOT}{n}{}{Allocate heap}{Allocates {\em n} bytes of heap
    space.  The space allocated is rounded to the next higher multiple
    of 4.}{}
\wline{AND}{n1 n2}{n3}{Bitwise AND}{Stores the bitwise AND of {\em n1}
    and {\em n2} on the stack.}{}
\wline{ARRAY {\em x}}{s$_1$ s$_2$ \ldots\ s$_n$ n esize}{}{Declare
    array}{Declares an array {\em x} of elements of {\em esize}
    bytes each with {\em n} subscripts, each ranging from 0 to {\em
    s}$_n-1$}{ARRAY}
\wline{ASIN}{f1}{f2}{${\mf f2}=\arcsin {\mf f1}$}{Replaces floating
    point top of stack with its arc sine.}{MATH}
\wline{ATAN}{f1}{f2}{${\mf f2}=\arctan {\mf f1}$}{Replaces floating
    point top of stack with its arc tangent.}{MATH}
\wline{ATAN2}{f1 f2}{f3}{${\mf f3}=\arctan {\mf f1}/{\mf f2}$}{Replaces
    the two floating point numbers on the top of the stack with
    the arc tangent of their quotient, properly handling zero
    denominators.}{MATH}
\wline{BEGIN}{}{}{Begin loop}{Begins an indefinite loop.  The end of
    the loop is marked by the matching {\tt AGAIN}, {\tt REPEAT}, or
    {\tt UNTIL}.}{}
\wline{BODY>}{pfa}{cfa}{Body to word}{Given body address of word,
    return the compile address of the word.}{DEFFIELDS}
\wline{>BODY}{cfa}{pfa}{Body address}{Given the compile address of a
    word, return its body (parameter) address.}{}
\wline{BRANCH}{}{}{Branch}{Jump to the address that follows in
    line.}{}
\wline{?BRANCH}{flag}{}{Conditional branch}{If the top of stack is
    zero, jump to the address which follows in line.  Otherwise 
    skip the address and continue execution.}{}
\wline{C!}{n addr}{}{Store byte}{The 8 bit value {\em n} is stored in
    the byte at address {\em addr}.}{}
\wline{C@}{addr}{n}{Load byte}{The byte at address {\em addr} is
    placed on the top of the stack.}{}
\wline{C,}{n}{}{Compile byte}{The 8 bit value {\em n} is stored in the
    next free byte of the heap and the heap pointer is incremented by
    one.}{}
\wline{C=}{}{}{Align heap}{The heap allocation pointer is adjusted to
    the next four byte boundary.  This must be done following a
    sequence of {\tt C,} operations.}{}
\wline{CLEAR}{}{}{Clear stack}{All items on the stack are
    discarded.}{}
\wline{COMPARE}{s1 s2}{n}{Compare strings}{The two strings whose
    addresses are given by {\em s1} and {\em s2} are compared.
    If {\em s1} is less than {\em s2}, $-1$ is returned; if
    {\em s1} is greater than {\em s2}, 1 is returned.  If
    {\em s1} and {\em s2} are equal, 0 is returned.}{STRING}
\wline{COMPILE {\em w}}{}{}{Compile word}{Adds the compile address of
   the word that follows in line to the definition currently being
   compiled.}{COMPILERW}
\wline{[COMPILE] {\em word}}{}{}{Compile immediate word}{Compiles the
    address of {\em word}, even if {\em word} is marked {\tt
    IMMEDIATE}.}{COMPILERW}
\wline{CONSTANT {\em x}}{n}{}{Declare constant}{Declares a constant
    named {\em x}.  When {\em x} is executed, the value {\em n} will
    be left on the stack.}{}
\wline{COS}{f1}{f2}{Cosine}{The floating point value on the top of the stack is
    replaced by its cosine.}{MATH}
\wline{CR}{}{}{Carriage return}{The standard output stream is advanced
    to the first character of the next line.}{CONIO}
\wline{CREATE}{}{}{Create object}{Create an object, given the name
    which appears next in the input stream, with a default action of
    pushing the parameter field address of the object when executed.
    No storage is allocated; normally the parameter field will be
    allocated and initialised by the defining word code that follows
    the {\tt CREATE}.}{}
\wline{DEPTH}{}{n}{Stack depth}{Returns the number of items on the
    stack before {\tt DEPTH} was executed.}{}
\wline{DO}{limit n}{}{Definite loop}{Executes the loop from the
    following word to the matching {\tt LOOP} or {\tt +LOOP} until
    {\em n} increments past the boundary between {\em limit}$-1$ and
    {\em limit}.  Note that the loop is always executed at least once
    (see {\tt ?DO} for an alternative to this).}{}
\wline{?DO}{limit n}{}{Conditional loop}{If {\em n} equals {\em
    limit}, skip immediately to the matching {\tt LOOP} or {\tt
    +LOOP}.  Otherwise, enter the loop, which is thenceforth treated
    as a normal {\tt DO} loop.}{}
\wline{DOES>}{}{}{Run-time action}{Sets the run-time action of a word
    created by the last {\tt CREATE} to the code that follows.  When
    the word is executed, its body address is pushed on the
    stack, then the code that follows the {\tt DOES>} will be
    executed.}{}
\wline{DROP}{n}{}{Discard top of stack}{Discards the value at the top
    of the stack.}{}
\wline{DUP}{n}{n n}{Duplicate}{Duplicates the value at the top of the
    stack.}{}
\wline{?DUP}{n}{0 / n n}{Conditional duplicate}{If top of stack is
    nonzero, duplicate it.  Otherwise leave zero on top of stack.}{}
\wline{ELSE}{}{}{Else}{Used in an {\tt IF}---{\tt ELSE}---{\tt THEN}
    sequence, delimits the code to be executed if the if-condition
    was false.}{}
\wline{EXECUTE}{addr}{}{Execute word}{Executes the word with compile
    address {\em addr}.}{}
\wline{EXIT}{}{}{Exit definition}{Exit from the current definition
   immediately.  Note that {\tt EXIT} cannot be used within
   a {\tt DO}---{\tt LOOP}; use {\tt LEAVE} instead.}{}
\wline{EXP}{f1}{f2}{${\mf f2}={\mf e}^{\mf f1}$}{The floating point value on
   the top of the stack is replaced by its natural
   antilogarithm.}{MATH}
\wline{F+}{f1 f2}{f3}{${\mf f3}={\mf f1}+{\mf f2}$}{The two
    floating point values on the top of the stack are added and their
    sum is placed on the top of the stack.}{REAL}
\wline{F-}{f1 f2}{f3}{${\mf f3}={\mf f1}-{\mf f2}$}{The
    floating point value {\em f2} is subtracted from the floating
    point value {\em f1} and the result
    is placed on the top of the stack.}{REAL}
\wline{F*}{f1 f2}{f3}{${\mf f3}={\mf f1}\times{\mf f2}$}{The two
    floating point values on the top of the stack are multiplied and their
    product is placed on the top of the stack.}{REAL}
\wline{F/}{f1 f2}{f3}{${\mf f3}={\mf f1}\div{\mf f2}$}{The
    floating point value {\em f1} is divided by the floating point
    value {\em f2} and the
    quotient is placed on the top of the stack.}{REAL}
\wline{F.}{f}{}{Print floating point}{The floating point value on the
    top of the stack is printed.}{REAL}
\wline{F<}{f1 f2}{flag}{Floating less than}{The top of stack is set to
    $-1$ if {\em f1} is less than
    {\em f2} and 0 otherwise.}{REAL}
\wline{F<=}{f1 f2}{flag}{Floating less than or equal}{The top of stack is
    set to
    $-1$ if {\em f1} is less than or equal to
    {\em f2} and 0 otherwise.}{REAL}
\wline{F<>}{f1 f2}{flag}{Floating not equal}{The top of stack is set to
    $-1$ if {\em f1} is not equal to
    {\em f2} and 0 otherwise.}{REAL}
\wline{F=}{f1 f2}{flag}{Floating equal}{The top of stack is set to
    $-1$ if {\em f1} is equal to
    {\em f2} and 0 otherwise.}{REAL}
\wline{F>}{f1 f2}{flag}{Floating greater than}{The top of stack is set to
    $-1$ if {\em f1} is greater than
    {\em f2} and 0 otherwise.}{REAL}
\wline{F>=}{f1 f2}{flag}{Floating greater than or equal}{The top of stack
    is set to
    $-1$ if {\em f1} is greater than or equal to
    {\em f2} and 0 otherwise.}{REAL}
\wline{FABS}{f1}{f2}{${\mf f2}=|{\mf f1}|$}{Replaces floating point
    top of stack with its absolute value.}{}
\wline{FCLOSE}{file}{}{Close file}{The specified file is
    closed.}{FILEIO}
\wline{FDELETE}{s1}{flag}{Delete file}{The file named by the string
    {\em s1} is deleted.  If the file was successfully deleted, $-1$
    is returned.   Otherwise, 0 is returned.}{FILEIO}
\wline{FGETC}{file}{char}{Read next character}{The next byte is read
    from the specified {\em file} and placed on the top of the stack.
    If end of file is encountered, $-1$ is returned.}{FILEIO}
\wline{FGETS}{file string}{flag}{Read string}{The next text line
    (limited to a maximum of 132 characters) is read from {\em file}
    and stored into the buffer at {\em string}.  Input lines are
    recognised in all the end of line conventions accepted by
    AutoCAD\@.  The end of line delimiter is deleted from the input
    line and is not stored in the {\em string}.
    If end of file is encountered 0 is returned; otherwise
    $-1$ is placed on the top of the stack.}{FILEIO}
\wline{FILE {\em f}}{}{}{Declare file}{A file descriptor named {\em f}
    is declared.  This descriptor may subsequently be associated
    with a file with {\tt FOPEN}.}{FILEIO}
\wline{FIND}{s}{word flag}{Look up word}{The word with name given
    by the string {\em s} is looked up in the dictionary.  If a
    definition if not found, {\em word} will be left as the address
    of the string and {\em flag} will be set to zero.  If the word is
    present in the dictionary, its compilation address is placed on
    the stack, followed by a {\em flag} that is 1 if the word is
    marked for immediate execution and $-1$ otherwise.}{DEFFIELDS}
\wline{FIX}{f}{n}{Floating to integer}{The floating point number
    on the top of the stack is replaced by the integer obtained by
    truncating its fractional part.}{REAL}
\wline{(FLIT)}{}{f}{Push floating point literal}{Pushes the
    floating point literal that follows in line onto the top of
    the stack.}{REAL}
\wline{FLOAD}{file}{stat}{Load file}{The source program starting at
    the current position in {\em file} is loaded as if its text
    appeared at the current character position in the input stream.
    The status resulting from the evaluation is left on the stack,
    zero if normal, negative in case of error.}{FILEIO}
\wline{FLOAT}{n}{f}{Integer to floating}{The integer value on the top
    of the stack is replaced by the equivalent floating point
    value.}{REAL}
\wline{FMAX}{f1 f2}{f3}{Floating point maximum}{The greater of the two
    floating point values on the top of the stack is placed on the
    top of the stack.}{FLOAT}
\wline{FMIN}{f1 f2}{f3}{Floating point minimum}{The lesser of the two
    floating point values on the top of the stack is placed on the
    top of the stack.}{FLOAT}
\wline{FNEGATE}{f1}{f2}{${\mf f2} = -{\mf f1}$}{The negative of the
    floating point value on the top of the stack replaces the floating
    point value there.}{FLOAT}
\wline{FOPEN}{fname fmodes file}{flag}{File open}{The previously
    declared {\em file} is opened with the specified file name {\em
    fname} given by the string address on the stack in the mode
    given by {\em fmodes}.  The bits in {\em fmodes} are 1 for read,
    2 for write, 4 for binary, and 8 to create a new
    file.  If the file is opened successfully, $-1$ is returned;
    otherwise 0 is returned.  The Unix standard streams, {\tt STDIN},
    {\tt STDOUT}, and {\tt STDERR} are predefined and automatically
    opened.}{FILEIO}
\wline{FORGET {\em w}}{}{}{Forget word}{The most recent definition of
    word {\em w} is deleted, along with all words declared more
    recently than the named word.}{}
\wline{FPUTC}{char file}{stat}{Write character}{The character {\em
    char} is written to {\em file}.  If the character is written
    successfully, {\em char} is returned; otherwise $-1$ is
    returned.}{FILEIO}
\wline{FPUTS}{s file}{flag}{Write string}{The string {\em s} is
    written to {\em file}, followed by the end of line delimiter used
    on this system.  If the line is written successfully, $-1$ is
    returned; otherwise 0 is returned.}{FILEIO}
\wline{FREAD}{file len buf}{length}{Read file}{{\em Len} bytes are
   read into buffer {\em buf} from {\em file}.  The number of
   bytes actually read is returned on the top of the stack.}{FILEIO}
\wline{FSEEK}{offset base file}{}{Set file position}{The current
    position of {\em file} is set to {\em offset}, relative to the
    specified {\em base}: if 0, the beginning of the file; if 1, the
    current file position; if 2, the end of file.}{FILEIO}
\wline{FSTRFORM}{f format str}{}{Floating point edit}{Edits a floating
    point number {\em f} into string {\em str}, using the {\tt
    sprintf} format given by the string {\em format}.}{REAL}
\wline{FTELL}{file}{pos}{File position}{Returns the current byte
    position {\em pos} for file {\em file}.}{FILEIO}
\wline{FWRITE}{len buf file}{length}{File write}{Writes {\em len}
    bytes from the buffer at address {\em buf} to {\em file}.
    The number of bytes written is returned on the top of the
    stack.}{FILEIO}
\wline{HERE}{}{addr}{Heap address}{The current heap allocation address
    is placed on the top of the stack.}{}
\wline{I}{}{n}{Inner loop index}{The index of the innermost {\tt
    DO}---{\tt LOOP} is placed on the stack.}{}
\wline{IF}{flag}{}{Conditional statement}{If {\em flag} is nonzero,
    the following statements are executed.  Otherwise, execution
    resumes after the matching {\tt ELSE} clause, if any, or after the
    matching {\tt THEN}.}{}
\wline{IMMEDIATE}{}{}{Mark immediate}{The most recently defined word
    is marked for immediate execution; it will be executed even if
    entered in compile state.}{}
\wline{J}{}{n}{Outer loop index}{The loop index of the next to
    innermost {\tt DO}---{\tt LOOP} is placed on the stack.}{}
\wline{L>NAME}{lfa}{nfa}{Link to name field}{Given the link field
    address of a word on the top of the stack, its name pointer field
    address is returned.}{DEFFIELDS}
\wline{LEAVE}{}{}{Exit {\tt DO}---{\tt LOOP}}{The innermost {\tt
    DO}---{\tt LOOP} is immediately exited.  Execution resumes after
    the {\tt LOOP} statement marking the end of the loop.}{}
\wline{LINK>}{lfa}{cfa}{Link field to compile address}{Given the link field
    address of a word on the top of the stack, the compile
    address of the word is returned.}{DEFFIELDS}
\wline{>LINK}{cfa}{lfa}{Link address}{Given the compile address of a
    word, return its link field address.}{DEFFIELDS}
\wline{(LIT)}{}{n}{Push literal}{Pushes the integer literal that
    follows in line onto the top of the stack.}{}
\wline{LITERAL}{n}{}{Compile literal}{Compiles the value on the top of
    the stack into the current definition.  When the definition is
    executed, that value will be pushed onto the top of the
    stack.}{COMPILERW}
\wline{LOG}{f1}{f2}{${\mf f2}=\ln {\mf f1}$}{The floating point value on
   the top of the stack is replaced by its natural logarithm.}{MATH}
\wline{LOOP}{}{}{Increment loop index}{Adds one to the index of 
    the active loop.  If the limit is reached, the loop is exited.
    Otherwise, another iteration is begun.}{}
\wline{+LOOP}{n}{}{Add to loop index}{Adds {\em n} to the index of 
    the active loop.  If the limit is reached, the loop is exited.
    Otherwise, another iteration is begun.}{}
\wline{<MARK}{}{addr}{Backward jump mark}{Saves the current
    compilation address on the stack.}{COMPILERW}
\wline{>MARK}{}{addr}{Forward mark}{Compiles a place-holder offset for
    a forward jump and saves its address for later backpatching on
    the stack.}{COMPILERW}
\wline{MAX}{n1 n2}{n3}{Maximum}{The greater of {\em n1} and {\em n2} is
    left on the top of the stack.}{}
\wline{MEMSTAT}{}{}{Print memory status}{The current and maximum
    memory usage so far are printed on standard output.  The sizes
    allocated for the stack, return stack, and heap are edited, as
    well as the percentage in use.}{MEMSTAT}
\wline{MIN}{n1 n2}{n3}{Minimum}{The lesser of {\em n1} and {\em n2} is
    left on the top of the stack.}{}
\wline{MOD}{n1 n2}{n3}{Modulus (remainder)}{The remainder when {\em
    n1} is divided by {\em n2} is left on the top of the stack.}{}
\wline{/MOD}{n1 n2}{n3 n4}{n3 = n1 $\bmod$ n2, n4 = n1 $\div$ n2}{
    Divides {\em n1} by {\em n2} and leaves quotient on top of
    stack, remainder as next on stack.}{}
\wline{N>LINK}{nfa}{lfa}{Name to link field}{Given the name field
    pointer address of a word on the top of the stack, leaves the link
    field address of the word on the top of stack.}{DEFFIELDS}
\wline{>NAME}{cfa}{nfa}{Name address}{Given the compile address of a
    word, return its name pointer field address.}{DEFFIELDS}
\wline{NAME>}{nfa}{cfa}{Name field to compile address}{Given the
    address of the name pointer field of a word on the top of the
    stack, leaves the compile address of the word on the top of the
    stack.}{DEFFIELDS}
\wline{NAME>S!}{nfa string}{}{Get name field}{Stores the
    name field of the word pointed to by {\em nfa} into {\em
    string}.}{DEFFIELDS} 
\wline{NEGATE}{n1}{n2}{${\mf n2}=-{\mf n1}$}{Negates the value on the
    top of the stack.}{}
\wline{(NEST)}{}{}{Invoke word}{Pushes the instruction pointer onto
    the return stack and sets the instruction pointer to the next word
    in line.}{}
\wline{NOT}{n1}{n2}{Logical not}{Inverts the bits in the value on the
    top of the stack.  This performs logical negation for truth
    values of $-1$ (True) and $0$ (False).}{}
\wline{OR}{n1 n2}{n3}{Bitwise OR}{Stores the bitwise OR of {\em n1}
    and {\em n2} on the stack.}{}
\wline{OVER}{n1 n2}{n1 n2 n1}{Duplicate second item}{The second item
    on the stack is copied to the top.}{}
\wline{PICK}{\ldots n$_2$ n$_1$ n$_0$ index}{\ldots n$_0$ n$_{index}$}{Pick
    item
    from stack}{The {\em index}th stack item is copied to the top of
    the stack.  The top of stack has {\em index} 0, the second item
    {\em index} 1, and so on.}{}
\wline{POW}{f1 f2}{f3}{${\mf f3}={\mf f1}^{\mf f2}$}{The second
    floating point value on the stack is taken to the power of the
    top floating point stack value and the result is left on the
    top of the stack.}{MATH}
\wline{QUIT}{}{}{Quit execution}{The return stack is cleared and
    control is returned to the interpreter.  The stack is not
    disturbed.}{}
\wline{>R}{n}{}{To return stack}{Removes the top item from the stack
    and pushes it onto the return stack.}{}
\wline{R>}{}{n}{From return stack}{The top value is removed from the
    return stack and pushed onto the stack.}{}
\wline{R@}{}{n}{Fetch return stack}{The top value on the return stack
    is pushed onto the stack.  The value is not removed from the
    return stack.}{}
\wline{REPEAT}{}{}{Close {\tt BEGIN}---{\tt WHILE}---{\tt REPEAT}
    loop}{Another iteration of the current {\tt BEGIN}---{\tt
    WHILE}---{\tt REPEAT} loop having been completed, execution
    continues after the matching {\tt BEGIN}.}{}
\wline{<RESOLVE}{addr}{}{Backward jump resolve}{Compiles the address
    saved by the matching {\tt <MARK}.}{COMPILERW}
\wline{>RESOLVE}{addr}{}{Forward jump resolve}{Backpatches the address left
    by the matching {\tt >MARK} to jump to the next word to be
    compiled.}{COMPILERW}
\wline{ROLL}{\ldots n$_2$ n$_1$ n$_0$ index}{\ldots n$_0$
    n$_{index}$}{Rotate {\em index}th item to top}{The stack item selected
    by {\em index}, with 0 designating the top of stack, 1 the
    second item, and so on, is moved to the top of the stack.  The
    intervening stack items are moved down one item.}{}
\wline{ROT}{n1 n2 n3}{n2 n3 n1}{Rotate 3 items}{The third item on the
    stack is placed on the top of the stack and the second and first
    items are moved down.}{}
\wline{-ROT}{n1 n2 n3}{n3 n1 n2}{Reverse rotate}{Moves the top of
    stack to the third item, moving the third and second items up.}{}
\wline{S!}{s1 s2}{}{Store string}{The string at address {\em s1} is
    copied into the string at {\em s2}.}{STRING}
\wline{S+}{s1 s2}{}{String concatenate}{The string at address {\em s1}
    is concatenated to the string at address {\em s2}.}{STRING}
\wline{S>NAME!}{string nfa}{}{Store name field}{Stores the {\em string}
    into the name field of the word given by name pointer
    field {\em nfa}.}{DEFFIELDS}
\wline{SHIFT}{n1 n2}{n3}{Shift n1 by n2 bits}{The value {\em n1} is
    logically shifted the number of bits specified by {\em n2},
    left if {\em n2} is positive and right if {\em n2} is
    negative.  Zero bits are shifted into vacated bits.}{}
\wline{SIN}{f1}{f2}{Sine}{The floating point value on the top of the stack is
    replaced by its sine.}{MATH}
\wline{SQRT}{f1}{f2}{${\mf f2}=\sqrt{{\mf f1}}$}{The floating point
    value on the top of the stack is replaced by its square
    root.}{MATH}
\wline{STATE}{}{addr}{System state variable}{The address of the system
    state variable is pushed on the stack.  The state is zero if
    interpreting, nonzero if compiling.}{}
\wline{STRCAT}{s1 s2}{}{String concatenate}{The string at address {\em s1}
    is concatenated to the string at address {\em s2}.}{STRING}
\wline{STRCHAR}{s1 s2}{}{String character search}{The string at
    address {\em s1} is searched for the first occurrence of the first
    character of string {\em s2}.  If that character appears nowhere
    in {\em s1}, 0 is returned.  Otherwise, the address of the first
    occurrence in {\em s1} is left on the top of the stack.}{STRING}
\wline{STRCMP}{s1 s2}{n}{String compare}{The string at address {\em s1}
    is compared to the string at address {\em s2}.  If {\em s1} is
    less than {\em s2}, $-1$ is returned.  If {\em s1} and {\em s2}
    are equal, 0 is returned.  If {\em s1} is greater than {\em s2}, 1
    is returned.}{STRING}
\wline{STRCPY}{s1 s2}{}{Store string}{The string at address {\em s1} is
    copied into the string at {\em s2}.}{STRING}
\wline{STRFORM}{n format str}{}{Integer edit}{Edits the
    number {\em n} into string {\em str}, using the {\tt
    sprintf} format given by the string {\em format}.  Note: the
    reference to the number in the format must be as a {\tt long}
    value, for example {\tt "\%ld"}.}{STRING}
\wline{STRING {\em x}}{size}{}{Declare string}{Declares a string named
    {\em x} of a maximum of {\em size}$-1$ characters.}{STRING}
\wline{STRINT}{s1}{s2 n}{String to integer}{Scans an integer
    from {\em s1}.  The integer scanned is placed on the top of
    the stack and the address of the character that terminated the
    scan is stored as the next item on the stack.}{STRING}
\wline{STRLEN}{s}{n}{String length}{The length of string {\em s} is
    placed on the top of the stack.}{STRING}
\wline{(STRLIT)}{}{s}{String literal}{Pushes the address of the string literal
    that follows in line onto the stack.}{STRING}
\wline{STRREAL}{s1}{s2 f}{String to real}{Scans a floating point
    number from {\em s1}.  The floating point number scanned is placed
    on the top of the stack and the address of the character that
    terminated the scan is stored as the next item on the
    stack.}{STRING}
\wline{SUBSTR}{s1 start length s2}{}{Extract substring}{The substring
    of string {\em s1} that begins at character {\em start}, with the
    first character numbered 0, extending for {\em length} characters,
    with $-1$ designating all characters to the end of string, is stored
    into the string {\em s2}.}{STRING}
\wline{SWAP}{n1 n2}{n2 n1}{Swap top two items}{The top two stack items
    are interchanged.}{}
\wline{SYSTEM}{s}{n}{Execute system command}{The operating system
    command given in the string {\em s} is passed to the system's
    command interpreter (shell).  The system result status returned
    after the command completes is left on the top of the
    stack.}{SYSTEM}
\wline{TAN}{f1}{f2}{Tangent}{The floating point value on the top of
    the stack is replaced by its tangent.}{MATH}
\wline{THEN}{}{}{End if}{Used in an {\tt IF}---{\tt ELSE}---{\tt THEN}
    sequence, marks the end of the conditional statement.}{}
\wline{TRACE}{n}{}{Trace mode}{If {\em n} is nonzero, trace mode is
    enabled.  If {\em n} is zero, trace mode is turned off.}{TRACE}
\wline{TYPE}{s}{}{Print string}{The string at address {\em s} is
    printed on standard output.}{CONIO}
\wline{UNTIL}{flag}{}{End {\tt BEGIN}---{\tt UNTIL} loop}{If {\em
    flag} is zero, the loop continues execution at the word following
    the matching {\tt BEGIN}\@.  If {\em flag} is nonzero, the loop is
    exited and the word following the {\tt UNTIL} is executed.}{}
\wline{VARIABLE {\em x}}{}{}{Declare variable}{A variable named {\em
    x} is declared and its value is set to zero.  When {\em x}
    is executed, its address will be placed on the stack.  Four bytes
    are reserved on the heap for the variable's value.}{}
\wline{WALKBACK}{n}{}{Walkback mode}{If {\em n} is nonzero, a walkback
    trace through active words will be performed whenever an error
    occurs during execution.  If {\em n} is zero, the walkback is
    suppressed.}{WALKBACK}
\wline{WHILE}{flag}{}{Decide {\tt BEGIN}---{\tt WHILE}---{\tt REPEAT}
    loop}{If {\em flag} is nonzero, execution continues after the {\tt
    WHILE}\@.  If {\em flag} is zero, the loop is exited and execution
    resumed after the {\tt REPEAT} that marks the end of the loop.}{}
\wline{WORDS}{}{}{List words defined}{Defined words are listed, from
    the most recently defined to the first defined.  If the system
    supports keystroke trapping, pressing any key will pause the
    display of defined words; pressing carriage return will abort the
    listing---any other key resumes it.  On other systems, only the 20
    most recently defined words are listed.}{CONIO}
\wline{WORDSUSED}{}{}{List words used}{The words used by this program
    are listed on standard output.  If the system supports keystroke
    trapping, the listing may be aborted by pressing a key while the
    output is in progress.  The words used report is useful in
    configuring a custom version of \atlast\ that includes just the words
    needed by the program it executes.}{WORDSUSED}
\wline{WORDSUNUSED}{}{}{List words not used}{The words not used by
    this program are listed on standard output.  If the system
    supports keystroke trapping, the listing may be aborted by
    pressing a key while the output is in progress.  The words not used
    report is useful in configuring a custom version of \atlast\ that
    includes just the words needed by the program it
    executes.}{WORDSUSED}
\wline{XOR}{n1 n2}{n3}{Bitwise exclusive OR}{Stores the bitwise
    exclusive or of {\em n1} and {\em n2} on the stack.}{}
\wline{(XDO)}{limit n}{}{Execute loop}{At runtime, enters a loop that
    will step until {\em n} increments and becomes equal to {\em limit}}{}
\wline{(X?DO)}{limit n}{}{Execute conditional loop}{At runtime, tests
    if {\em n} equals {\em limit}.  If so, skips until the matching {\tt
    LOOP} or {\tt +LOOP}.  Otherwise, enters the loop.}{}
\wline{(XLOOP)}{}{}{Increment loop index}{At runtime, adds one to the
    index of the active loop and exits if equal to the limit.
    Otherwise returns to the matching {\tt DO} or {\tt ?DO}.}{}
\wline{(+XLOOP)}{incr}{}{Add to loop index}{At runtime, increments
    the loop index by the top of stack.  If the loop is not done, begins
    the next iteration.}{}

\end{tabbing}

\clearpage

\runhead{\huge Benchmark Program Listings}

\centerline{\fbox{\Large \tt SQRT.ATL}}

\begin{verbatim}
2variable x
2variable y

: csqrt
        2dup 0.0 f< if
           cr ." "SQRT: Negative argument!"
           exit
        then
        2dup 0.0 f<> if
           2dup 2dup x 2!
           1.893872 f* 0.154116 f+
           1.047988 f* 1.0 f+
           f/ y 2!                 \ y=(0.154116+1.893872*x)/(1.0+1.047988*x)

           y 2@                    \ y
           0.0                     \ y c
           begin
                 2swap             \ c y
                 2dup              \ c y y
                 x 2@              \ c y y x
                 2over             \ c y y x y
                 f/                \ c y y x/y
                 f-                \ c y y-x/y
                 -0.5              \ c y (y-x/y) -0.5
                 f*                \ c y (y-x/y)*-0.5
                 2dup              \ cl y c c
                 2rot              \ cl c c y
                 f+                \ cl c c+y
                 2rot              \ c c+y cl
                 2rot              \ c+y cl c
                 2swap             \ c+y c cl
                 2over             \ c+y c cl c
                 f=                \ c+y c =0?
           until
           2drop
        then
;

: cbenchmark 10000 0 do 2.0 csqrt 2drop loop ." "Done\n" ;
: sbenchmark 100000 0 do 2.0 sqrt 2drop loop ." "Done\n" ;

.( "Type \"cbenchmark\" to run the CSQRT benchmark (10000 iterations).\n"
.( "Type \"sbenchmark\" to run the SQRT benchmark (100000 iterations).\n"
\end{verbatim}
\clearpage
\centerline{\fbox{\Large \tt CSQRT.C}}

\begin{verbatim}
#include <stdio.h>

double
/*FCN*/asqrt(x)
  double x;
{
    double c, cl, y;
    int n;

    if (c == 0.0)
        return (0.0);

    if (x < 0.0)
        abort();

    y = (0.154116 + 1.893872 * x) / (1.0 + 1.047988 * x);
    c = 0.0;
    n = 20;
    do {
        cl = c;
        c = (y - x / y) * 0.5;
        y -=  c;
    } while (c != cl && --n);
    return y;
}

main()
{
    int i;
    char a[300];

    fputs("Ready to test: ", stdout);
    gets(a);

    for (i = 0; i < 100000; i++)
        asqrt(2.0);
    printf("Done.\n");
}
\end{verbatim}
\clearpage
\centerline{\fbox{\Large \tt SSQRT.C}}

\begin{verbatim}
#include <stdio.h>
#include <math.h>

main()
{
    int i;
    char a[300];

    fputs("Ready to test: ", stdout);
    gets(a);

    for (i = 0; i < 100000; i++)
        sqrt(2.0);
    printf("Done.\n");
}
\end{verbatim}
\clearpage
\centerline{\fbox{\Large \tt SQRT.LSP}}

\begin{verbatim}
(defun sqr (x / y c cl)
    (if (or (= 'REAL (type x)) (= 'INT (type x)))
        (progn
            (cond ((minusp x) 'Negative-argument)
                  ((zerop x) 0.0)
                  (t (setq y (/ (+ 0.154116 (* x 1.893872))
                                       (+ 1.0 (* x 1.047988))
                             )
                     )
                     (setq c (/ (- y (/ x y)) 2.0))
                     (setq cl 0.0)
                     (while (not (equal c cl))
                        (setq y (- y c))
                        (setq cl c)
                        (setq c (/ (- y (/ x y)) 2.0))
                     )
                     y
                  )
            )
        )
        (progn
            (princ "Invalid argument.")
            (princ)
        )
    )
)

(defun C:csqrt () (repeat 10000 (sqr 2.0)))
(defun C:ssqrt () (repeat 10000 (sqrt 2.0)))
\end{verbatim}


\end{document}
